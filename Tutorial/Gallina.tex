




Gallina is the specification language of Coq. 
It is a functional language, fairly similar to \href{https://en.wikipedia.org/wiki/OCaml}{OCaml} or 
\href{https://en.wikipedia.org/wiki/Standard_ML}{SML} if you are familiar with those. 
It includes things such as pattern matching, let-in definitions, and recursive functions. 

~\\
The reference manual for the Gallina Language Specification can be found here: 
\\ 
\url{https://coq.inria.fr/distrib/current/refman/language/gallina-specification-language.html#}

~\\ 
This details the grammar of the language; 
lexical conventions (i.e. keywords, formatting of identifiers/variables, etc.); 
the syntax of terms; types; etc; then goes into the grammar of The Vernacular (the language of commands of Gallina). 
If you are familiar with programming language basics like these, you may find this of interest 
and it could be beneficial to you; if not, it may be a bit confusing 

~\\
A few important things to note about the Vernacular of Gallina are that each sentence begins with a capital letter and ends 
with a dot (i.e. period), and whitespace is used to separate terms but otherwise ignored.

~\\ 
Starting in the 
\href{https://coq.inria.fr/distrib/current/refman/language/gallina-specification-language.html#assumptions}
{Assumptions}
section, there are some examples mixed in with the formal command definitions 
and syntax specifications for their use, which may be beneficial if you are struggling with a particular command 
and need more information beyond what this tutorial provides. 











