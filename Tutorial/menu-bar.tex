
\paragraph{\underline{File}}
~\\
\TT{New} : Open a new blank scratch buffer (for writing definitions and proving in Coq)
\\
\TT{Open} : Open a file stored in memory
\\
\TT{Save} : Save your current file. Functions as \TT{Save as} when used on an unsaved scratch buffer. 
\\ 
\TT{Save as} : Opens a pop-up window to choose where to save the current buffer and give it a name. 
\\
\TT{Revert all buffers} : ? % Seems to have no effect when clicked.
\\
\TT{Close buffer} : Close the current buffer. Will give a warning if you have unsaved changes. 
\\
\TT{Print} : Shows you the command to use in the command line to print the current buffer file. 
\\
\TT{Export to} : Allows you to export the current file as a file of a different type. 
	\\ \-\ \qquad Options: Html, LaTeX, Dvi, Pdf, Ps
\\
\TT{Rehighlight} : ? % Seems to have no effect when clicked.
\\
\TT{Quit} : Close the CoqIDE. Will give a warning if there are unsaved changes. 





~\\
\paragraph{\underline{Edit}}
~\\
\TT{Undo} : Undo last typing action
\\
\TT{Redo} : Redo last typing action removed by \TT{Undo}
\\
\TT{Cut} : Cut text. Can be pasted into other programs. 
\\
\TT{Copy} : Copy text. Can be pasted into other programs.
\\ 
\TT{Paste} : Paste copied text. Works with text copied from other programs, but can lose characters (i.e., \_)
\\
\TT{Find / Replace} : Opens a window to find text (and replace if desired). 
	Can be detached to be in its own window separate from the main CoqIDE. 
\\
\TT{Find Next} : ? 
\\
\TT{Find Previous} : ? 
\\
\TT{External editor} : Looks for and attempts to load a \TT{.aux} file of the same name as the current buffer. 
	Prints results in \TT{Message} window. 
\\
\TT{Preferences} : Pop up window to modify your preferences for the CoqIDE 
(i.e., customize font, font size, editor configurations, colors, etc.)





~\\
\paragraph{\underline{View}}
~\\
\TT{Previous tab} : 
\\
\TT{Next tab} : 
\\
\TT{Zoom in} : 
\\
\TT{Zoom out} : 
\\
\TT{Zoom fit} : 
\\
\TT{Show Toolbar} : 
\\
\TT{Query Pane} : 
\\
\TT{Display implicit arguments} : 
\\
\TT{Display coercions} : 
\\
\TT{Display raw matching expressions} : 
\\
\TT{Display notations} : 
\\
\TT{Display all basic low-level contents} : 
\\
\TT{Display existential variable instances} : 
\\
\TT{Display universe levels} : 
\\
\TT{Display all low-level contents} : 
\\
\TT{Display unfocused goals} : 
\\
\TT{Don't show diffs}, \TT{Show diffs:only added}, \TT{Show diffs:added and removed} : Choices for displaying diffs (must choose from one of the three)

	
	
	
~\\
\paragraph{\underline{Navigation}}
~\\
Note: these are like their toolbar button equivalents. 
Please see the subsection \ref{subsec: toolbar} for further descriptions given there. 
\\
\TT{Forward} : Forward one command. 
\\ 
\TT{Backward} : Backward one command. 
\\
\TT{Go to} : Go to cursor. 
\\
\TT{Start} : Restart Coq. 
\\
\TT{End} : Go to end. 
\\
\TT{Interrupt} : Interrupt computations. 
\\
\TT{Previous} : Previous Occurrence. 
\\
\TT{Next} : Next Occurrence. 




	
~\\
\paragraph{\underline{Tactics}}
~\\




	
~\\
\paragraph{\underline{Templates}}
~\\




	
~\\
\paragraph{\underline{Queries}}
~\\
These are described and have examples of their use in section \ref{Sec: queries}. 
\\
\TT{Search} : See subsection \ref{search}. 
\\
\TT{Check} : See subsection \ref{check}. 
\\
\TT{Print} : See subsection \ref{print}. 
\\
\TT{About} : See subsection \ref{about}. 
\\
\TT{Locate} : See subsection \ref{locate}. 
\\
\TT{Print Assumptions} : See subsection \ref{print_assumptions}. 



	
~\\
\paragraph{\underline{Tools}}
~\\




	
~\\
\paragraph{\underline{Compile}}
~\\




	
~\\
\paragraph{\underline{Windows}}
~\\





	
~\\
\paragraph{\underline{Help}}
~\\
\TT{Browse Coq Manual} : Opens \url{https://coq.inria.fr/distrib/V8.10.2/refman/}
\\ 
\TT{Browse Coq Library} : Opens \url{https://coq.inria.fr/distrib/V8.10.2/stdlib/}
\\
\TT{Help for keyword} : ? % nothing seems to happen
\\
\TT{Help for $\mu$PG mode} : Prints out help for this mode (for use with Emacs, I believe) in the \TT{Message} window.
\\ 
\TT{About} : Pop-up window giving info about current CoqIDE.











