
\paragraph{\underline{File}}
~\\
\TT{New} : Open a new blank scratch buffer (for writing definitions and proving in Coq)
\\
\TT{Open} : Open a file stored in memory
\\
\TT{Save} : Save your current file. Functions as \TT{Save as} when used on an unsaved scratch buffer. 
\\ 
\TT{Save as} : Opens a pop-up window to choose where to save the current buffer and give it a name. 
\\
\TT{Revert all buffers} : Likely intended to undo changes in all buffers, back to previously saved state 
	(Currently seems to have no effect).  % ? Seems to have no effect when clicked.
\\
\TT{Close buffer} : Close the current buffer. Will give a warning if you have unsaved changes. 
\\
\TT{Print} : Shows you the command to use in the command line to print the current buffer file. 
\\
\TT{Export to} : Allows you to export the current file as a file of a different type. 
	\\ \-\ \qquad Options: Html, LaTeX, Dvi, Pdf, Ps
\\
\TT{Rehighlight} : Rehighlight text. % ? Seems to have no effect when clicked.
\\
\TT{Quit} : Close the CoqIDE. Will give a warning if there are unsaved changes. 





~\\
\paragraph{\underline{Edit}}
~\\
\TT{Undo} : Undo last typing action
\\
\TT{Redo} : Redo last typing action removed by \TT{Undo}
\\
\TT{Cut} : Cut text. Can be pasted into other programs. 
\\
\TT{Copy} : Copy text. Can be pasted into other programs.
\\ 
\TT{Paste} : Paste copied text. Works with text copied from other programs, but can lose characters (i.e., \_)
\\
\TT{Find / Replace} : Opens a window to find text (and replace if desired). 
	Can be detached to be in its own window separate from the main CoqIDE. 
\\
\TT{Find Next} : Go to next occurrence.  %? Seemingly not working. 
\\
\TT{Find Previous} : Go to previous occurrence.  %? Seemingly not working. 
\\
\TT{External editor} : Looks for and attempts to load a \TT{.aux} file of the same name as the current buffer. 
	Prints results in \TT{Message} window. 
\\
\TT{Preferences} : Pop up window to modify your preferences for the CoqIDE 
(i.e., customize font, font size, editor configurations, colors, etc.)





~\\
\paragraph{\underline{View}}
~\\
\TT{Previous tab} : Move to the buffer in the tab before the current one (can also click on the tab of the desired buffer). 
\\
\TT{Next tab} : Move to the buffer in the tab after the current one (can also click on the tab of the desired buffer).
\\
\TT{Zoom in} : Increase the size of the text in the buffer, messages, and goal window (affects all buffers).  
\\
\TT{Zoom out} : Decrease the size of the text in the buffer, messages, and goal window (affects all buffers).  
\\
\TT{Zoom fit} : Intended to return to normal zoom level % ? Doesn't seem to have an affect currently.
\\
\TT{Show Toolbar} : Toggle to show/hide toolbar (toolbar discussed in subsection \ref{subsec: toolbar}). 
\\
\TT{Query Pane} : Shows/hides the query pane (appears across the bottom of the CoqIDE). 
	Can be detached into its own window. 
\\
\TT{Display implicit arguments} : 
\\
\TT{Display coercions} : 
\\
\TT{Display raw matching expressions} : 
\\
\TT{Display notations} : 
\\
\TT{Display all basic low-level contents} : 
\\
\TT{Display existential variable instances} : 
\\
\TT{Display universe levels} : 
\\
\TT{Display all low-level contents} : 
\\
\TT{Display unfocused goals} : 
\\
\TT{Don't show diffs}, \TT{Show diffs:only added}, \TT{Show diffs:added and removed} : Choices for displaying diffs (must choose from one of the three)

	
	
	
~\\
\paragraph{\underline{Navigation}}
~\\
Note: these are like their toolbar button equivalents. 
Please see the subsection \ref{subsec: toolbar} for further descriptions given there. 
\\
\TT{Forward} : Forward one command. 
\\ 
\TT{Backward} : Backward one command. 
\\
\TT{Go to} : Go to cursor. 
\\
\TT{Start} : Restart Coq. 
\\
\TT{End} : Go to end. 
\\
\TT{Interrupt} : Interrupt computations. 
\\
\TT{Previous} : Previous Occurrence. 
\\
\TT{Next} : Next Occurrence. 




	
~\\
\paragraph{\underline{Tactics}}
~\\
Places tactics that can be used in proofs at the cursor location in the current buffer. 
Some of these are discussed in further detail in Section \ref{Sec: tactics}; 
all can be found in the \href{https://coq.inria.fr/distrib/V8.10.2/refman/coq-tacindex.html}{online documentation}. 
~\\
\TT{a...} : Places the given tactic. Includes options: 
	\TT{abstract}, \TT{absurd}, \TT{apply}, \TT{apply \_ with}, \TT{assert}, \TT{assert (\_:\_)}, 
	\TT{assert (\_:=\_)}, \TT{assumption}, \TT{auto}, \TT{auto with}, \TT{autorewrite}. 
\\
\TT{c...} : Places the given tactic. Includes options: 
	\TT{case}, \TT{case \_ with}, \TT{casetype}, \TT{cbv}, \TT{cbv in}, \TT{change}, \TT{change \_ in}, 
	\TT{clear}, \TT{clearbody}, \TT{cofix}, \TT{compare}, \TT{compute}, \TT{compute in}, \TT{congruence}, 
	\TT{constructor}, \TT{constructor \_ with}, \TT{contradiction}, \TT{cut}, \TT{cutrewrite}. 
\\
\TT{d...} : Places the given tactic. Includes options: 
	\TT{decide equality}, \TT{decompose}, \TT{decompose record}, \TT{decompose sum}, 
	\TT{dependent inversion}, \TT{dependent inversion \_ with}, \TT{dependent inversion\_clear}, 
	\TT{dependent inversion\_clear \_ with}, \TT{dependent rewrite ->}, \TT{dependent rewrite <-}, 
	\TT{destruct}, \TT{discriminate}, \TT{do}, \TT{double induction}. 
\\
\TT{e...} : Places the given tactic. Includes options: 
	\TT{eapply}, \TT{eauto}, \TT{eauto with}, \TT{eexact}, \TT{elim}, \TT{elim \_ using}, 
	\TT{elim \_ with}, \TT{elimtype}, \TT{exact}, \TT{exists}. 
\\
\TT{f...} : Places the given tactic. Includes options: 
	\TT{fail}, \TT{field}, \TT{first}, \TT{firstorder}, \TT{firstorder using}, \TT{firstorder with}, 
	\TT{fix}, \TT{fix \_ with}, \TT{fold}, \TT{fold \_ int}, \TT{functional induction}. 
\\
\TT{g...} : Places the given tactic. Includes options: 
	\TT{generalize}, \TT{generalize dependent}. 
\\
\TT{hnf} : Replaces the current goal with its head normal form. 
\\
\TT{i...} : Places the given tactic. Includes options: 
	\TT{idtac}, \TT{induction}, \TT{info}, \TT{injection}, \TT{instantiate (\_:=\_)}, \TT{intro}, 
	\TT{intro after}, \TT{intro \_ after}, \TT{intros}, \TT{intros until}, \TT{intuition}, 
	\TT{inversion}, \TT{inversion \_ in}, \TT{inversion \_ using}, \TT{inversion \_ using \_ in}, 
	\TT{inversion\_clear}, \TT{inversion\_clear \_ in}.  
\\
\TT{j...} : Places the given tactic. Includes options: 
	\TT{jp <n>}, \TT{jp}. 
\\
\TT{l...} : Places the given tactic. Includes options: 
	\TT{lapply}, \TT{lazy}, \TT{lazy in}, \TT{left}. 
\\
\TT{move ... after} : Moves the hypothesis named in \TT{...} after the hypothesis given beyond 
	\TT{after} in the local context. 
\\
\TT{omega} : Automatic decision procedure for Presburger arithmetic. 
	Must be loaded using command \TT{Require Import Omega} before use. 
\\
\TT{p...} : Places the given tactic. Includes options: 
	\TT{pattern}, \TT{pose}, \TT{pose \_:=\_)}, \TT{progress}. 
\\
\TT{quote} : The quote plugin was removed, documentation no longer includes information on this tactic. 
\\
\TT{r...} : Places the given tactic. Includes options: 
	\TT{red}, \TT{red in}, \TT{refine}, \TT{reflexivity}, \TT{rename \_ into}, \TT{repeat}, 
	\TT{replace \_ with}, \TT{rewrite}, \TT{rewrite \_ in}, \TT{rewrite <-}, 
	\TT{rewrite <- \_ in}, \TT{right}, \TT{ring}. 
\\
\TT{s...} : Places the given tactic. Includes options: 
	\TT{set}, \TT{set (\_:=\_)}, \TT{setoid\_replace}, \TT{setoid\_rewrite}, \TT{simpl}, 
	\TT{simpl \_ in}, \TT{simple destruct}, \TT{simple induction}, \TT{simple inversion}, 
	\TT{simplify\_eq}, \TT{solve}, \TT{split}, \TT{subst}, \TT{symmetry}, \TT{symmetry in}. 
\\
\TT{t...} : Places the given tactic. Includes options: 
	\TT{tauto}, \TT{transivity}, \TT{trivial}, \TT{try}. 
\\
\TT{u...} : Places the given tactic. Includes options: 
	\TT{unfold}, \TT{unfold \_ in}.  






	
~\\
\paragraph{\underline{Templates}}
~\\
Places the selected item in the current buffer at the cursor's location. 
Gives the most commonly desired items at top, and others within multi-option menus. 
~\\
\TT{Lemma} : Places the template for a new lemma. 
\\
\TT{Theorem} : Places the template for a new theorem. 
\\
\TT{Definition} : Places the template for a new definition. 
\\
\TT{Inductive} : Places the template for a new inductive definition. 
\\
\TT{Fixpoint} : Places the template for a new fixpoint definition. 
\\
\TT{Scheme} : Places the template for a new scheme. 
\\
\TT{match} : Places the template for pattern matching using \TT{match}. 
	Requires this to be inside an inductive type (doesn't seem to be working properly). 
\\
\TT{A...} : Places the given text. Options: 
	\TT{Add Abstract Ring A Aplus Amult Aone Azero Ainv Aeq T.}, 
	\TT{Add Abstract Semi Ring A Aplus Amult Aone Azero Aeq T.}, 
	\TT{Add Field}, \TT{Add LoadPath}, \TT{Add ML Path}, \TT{Add Morphism}, 
	\TT{Add Printing Constructor}, \TT{Add Printing If}, \TT{Add Printing Let}, 
	\TT{Add Printing Record}, \TT{Add Rec LoadPath}, \TT{Add Rec ML Path}, 
	\TT{Add Ring A Aplus Amult Aone Azero Ainv Aeq T [ c1 ... cn ].  }, 
	\TT{Add Semi Ring A Aplus Amult Aone Azero Aeq T [ c1 ... cn ].}, 
	\TT{Add Relation}, \TT{Add Setoid}, \TT{Axiom}. 
\\
\TT{C...} : Places the given text. Options: 
	\TT{Canonical Structure}, \TT{Chapter}, \TT{Coercion}, \TT{Coercion Local}, 
	\TT{CoFixpoint}, \TT{CoInductive}. 
\\
\TT{D...} : Places the given text. Options: 
	\TT{Declare ML Module}, \TT{Defined.}, \TT{Definition}, \TT{Derive Dependent Inversion}, 
	\TT{Derive Dependent Inversion\_clear}, \TT{Derive Inversion}, \TT{Derive Inversion\_clear}.
\\
\TT{E...} : Places the given text. Options: 
	\TT{End}, \TT{End Silent.}, \TT{Eval}, \TT{Extract Constant}, \TT{Extract Inductive}, 
	\TT{Extraction Inline}, \TT{Extraction Language}, \TT{Extraction NoInline}. 
\\
\TT{F...} : Places the given text. Options: 
	\TT{Fact}, \TT{Fixpoint}, \TT{Focus}. 
\\
\TT{G...} : Places the given text. Options: 
	\TT{Global Variable}, \TT{Goal}, \TT{Grammar}. 
\\
\TT{H...} : Places the given text. Options: 
	\TT{Hint}, \TT{Hint Constructors}, \TT{Hint Extern}, \TT{Hint Immediate}, 
	\TT{Hint Resolve}, \TT{Hint Rewrite}, \TT{Hint Unfold}, \TT{Hypothesis}. 
\\
\TT{I...} : Places the given text. Options: 
	\TT{Identity Coercion}, \TT{Implicit Arguments}, \TT{Inductive}, \TT{Infix}. 
\\
\TT{L...} : Places the given text. Options: 
	\TT{Lemma}, \TT{Load}, \TT{Load Verbose}, \TT{Local}, \TT{Ltac}. 
\\
\TT{M...} : Places the given text. Options: 
	\TT{Module}, \TT{Module Type}, \TT{Mutual Inductive}. 
\\
\TT{N...} : Places the given text. Options: 
	\TT{Notation}, \TT{Next Obligation}. 
\\
\TT{O...} : Places the given text. Options: 
	\TT{Opaque}, \TT{Obligations Tactic}. 
\\
\TT{P...} : Places the given text. Options: 
	\TT{Parameter}, \TT{Proof.}, \TT{Program Definition}, \TT{Program Fixpoint}, 
	\TT{Program Lemma}, \TT{Program Theorem}. 
\\
\TT{Qed} : Places the \TT{Qed.} command for completing a proof. 
\\
\TT{R...} : Places the given text. Options: 
	\TT{Read Module}, \TT{Record}, \TT{Variant}, \TT{Remark}, \TT{Remove LoadPath}, 
	\TT{Remove Printing Constructor}, \TT{Remove Printing If}, \TT{Remove Printing Let}, 
	\TT{Remove Printing Record}, \TT{Require}, \TT{Require Export}, \TT{Require Import}, 
	\TT{Reset Extraction Inline}, \TT{Restore State}. 
\\
\TT{S...} : Places the given text. Options: 
	\TT{Scheme}, \TT{Section}, \TT{Set Extraction AutoInline}, \TT{Set Extraction Optimize}, 
	\TT{Set Hyps\_limit}, \TT{Set Implicit Arguments}, \TT{Set Printing Wildcard}, \TT{Set Silent.}, 
	\TT{Set Undo}, \TT{Structure}, \TT{Syntactic Definition}, \TT{Syntax}. 
\\
\TT{T...} : Places the given text. Options: 
	\TT{Test Printing If}, \TT{Test Printing Let}, \TT{Test Printing Synth}, \TT{Test Printing Wildcard}, 
	\TT{Theorem}, \TT{Time}, \TT{Transparent}. 
\\
\TT{U...} : Places the given text. Options: 
	\TT{Unfocus}, \TT{Unset Extraction AutoInline}, \TT{Unset Extraction Optimize}, 
	\TT{Unset Hyps\_limit}, \TT{Unset Implicit Arguments}, \TT{Unset Printing Wildcard}, 
	\TT{Unset Silent.}, \TT{Unset Undo}. 
\\
\TT{V...} : Places the given text. Options: 
	\TT{Variable}, \TT{Variables}. 
\\
\TT{Write State} : Places the given text. 





	
~\\
\paragraph{\underline{Queries}}
~\\
These are described and have examples of their use in section \ref{Sec: queries}. 
\\
\TT{Search} : See subsection \ref{search}. 
\\
\TT{Check} : See subsection \ref{check}. 
\\
\TT{Print} : See subsection \ref{print}. 
\\
\TT{About} : See subsection \ref{about}. 
\\
\TT{Locate} : See subsection \ref{locate}. 
\\
\TT{Print Assumptions} : See subsection \ref{print_assumptions}. 



	
~\\
\paragraph{\underline{Tools}}
~\\
\TT{Comment} : Comments out highlighted text. 
	Will wrap another commented out layer around currently commented out text. 
\\ 
\TT{Uncomment} : Uncomments highlighted text. If not commented out, no effect. 
\\
\TT{Coqtop arguments} : Opens a separate window with the Coq command for this. 
\\
\TT{LaTeX-to-unicode} : Likely used to convert text in file from LaTeX to unicode. 





	
~\\
\paragraph{\underline{Compile}}
~\\
\TT{Compile buffer} : Compiles the current buffer, creates a \TT{.glob} file and a \TT{.vo} file. 
\\
\TT{Make} : Looks for a makefile and runs it if found. 
	Compiles the buffers included in the makefile as the above command would, 
	and creates \TT{.aux} files for each. 
\\
\TT{Next error} : Go to next error, if there is one. % doesn't always seem to work
\\
\TT{Make makefile} : Creates a makefile (\TT{makefile}) for the current directory, 
	and a makefile configuration file (\TT{makefile.conf}). Can then be run using the \TT{Make} 
	command from above. Has a \TT{make clean} command to clean up after the make if desired. 
 




	
~\\
\paragraph{\underline{Windows}}
~\\
\TT{Detach view} : Pops out the goal window into its own separate window. 
	To move it back to the IDE window, close the detached window. 




	
~\\
\paragraph{\underline{Help}}
~\\
\TT{Browse Coq Manual} : Online reference manual. Opens \url{https://coq.inria.fr/distrib/V8.10.2/refman/}
\\ 
\TT{Browse Coq Library} : Online standard library. Opens \url{https://coq.inria.fr/distrib/V8.10.2/stdlib/}
\\
\TT{Help for keyword} : Searches for highlighted keyword 
	(Doesn't find the documentation well - websites are much more useful). 
\\
\TT{Help for $\mu$PG mode} : Prints out help for this mode (for use with Emacs, I believe) in the \TT{Message} window.
\\ 
\TT{About} : Pop-up window giving info about current CoqIDE.











