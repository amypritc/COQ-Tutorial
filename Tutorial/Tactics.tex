
This section contains brief descriptions of some commonly used proof tactics.
For the full breakdown of all Coq tactics, please refer to the Tactic Index of the Coq documentation that can be found here:
\url{https://coq.inria.fr/distrib/current/refman/coq-tacindex.html}.


\subsection{simpl} \label{simpl}
Simplify both sides of an equation.

\noindent
Examples using this tactic: 
%\ref{count_to_n},
\ref{sum_to_n}




\subsection{reflexivity} \label{reflexivity}
Check if both sides of an equation are equal. 
Will do more simplifications that simpl; tries unfolding and expanding definitions.

\noindent
Examples using this tactic: 
\ref{sum_to_n}




\subsection{trivial} \label{trivial}
A restriction of auto that is not recursive. 
Tries simple hints, like solving trivial equalities such as x=x.

\noindent
Examples using this tactic: 
%\ref{count_to_n},
\ref{sum_to_n}



\subsection{auto} \label{auto}
First attempts to use assumption, then uses intros and generates any hypotheses to use as hints and attempt to apply.

\noindent
Examples using this tactic: 



\subsection{ring} \label{ring}
Very useful when proving over numbers.
Solves equations upon polynomial expressions of a ring structure. 
Normalizes and compares results.
Uses properties like associativity, commutativity, distributivity, and constant propagation.

\noindent
Examples using this tactic: 
\ref{sum_to_n}


\subsection{discriminate} \label{discriminate}
Proves any goal in an assumption stating two structurally different terms of an inductive set are equal. For example, S (S O) = S O.

\noindent
Examples using this tactic: 



\subsection{assumption} \label{assumption}
Looks in the local context for the hypothesis. The type must be convertible to the goal.

\noindent
Examples using this tactic: 



\subsection{intros} \label{intros}
Introduces variables from the goal into the proof environment for use.

\noindent
Examples using this tactic: 
%\ref{count_to_n}, 
\ref{sum_to_n}


\subsection{contradiction} \label{contradiction}
Attempts to find a hypothesis equivalent to:
\begin{itemize}
	\item an empty inductive type (i.e. false)
	\item the negation of a single inductive type (i.e. true, x=x)
	\item two contradictory hypotheses
\end{itemize}

\noindent
Examples using this tactic: 



\subsection{induction} \label{induction}
The object must be of an inductive type to use induction. 
This generates subgoals and an induction hypothesis.

\noindent
Examples using this tactic: 
%\ref{count_to_n}



\subsection{functional induction} \label{functional induction}
Very useful for inductive/recursive functions.
Performs case analysis and induction on the definition of a function.
To use functional induction, you need to make sure to require and load FunInd, and then define the functional induction scheme.

\begin{code}
\cmd{Require} \nm{FunInd}.	\\
\Load FunInd.
\\ \\
Functional Scheme name\_ind := 
  Induction for name Sort \ty{Prop}.
\end{code}

\noindent
Examples using this tactic: 
\ref{sum_to_n}


\subsection{destruct} \label{destruct}
Creates subgoals from a more complex goal. 
Subgoals must be proven separately.
Can be used with any inductively defined type.
Can be used nested if a generated subgoal needs to be broken up further.
Allows you to specify the names of variables to be used in proving the subgoals.

\noindent
Examples using this tactic: 



\subsection{rewrite} \label{rewrite}
A way to apply previously defined assumptions.
Can use arrows $<-$ , $->$ to give directionality for the application.
Example \ref{sum_to_n} demonstrates the use of arrows to give directionality to rewrite.

\noindent
Examples using this tactic: 
%\ref{count_to_n}, 
\ref{sum_to_n}


\subsection{apply} \label{apply}
Attempts to match the current goal against the conclusion of the given term.

\noindent
Examples using this tactic:





%\subsection{} \label{}


