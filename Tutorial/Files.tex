

\subsection{Standard Libraries} \label{stdlib}

\begin{tabular}{l L}
Logic 	&
	Classical logic and dependent equality
\\	
Arith 		&
	Basic Peano arithmetic
\\
PArith 	&
	Basic positive integer arithmetic
\\
NArith  	&
	Basic binary natural number arithmetic
\\
ZArith 	&
	Basic relative integer arithmetic
\\
Numbers 	&
	Various approaches to natural, integer and cyclic numbers (currently axiomatically and on top of $2^{31}$ binary words)
\\
Bool 		&
	Booleans (basic functions and results)
\\
Lists�		& 
	Monomorphic and polymorphic lists (basic functions and results), Streams (infinite sequences defined with co-inductive types)
\\
Sets		& 
	Sets (classical, constructive, finite, infinite, power set, etc.)
\\
FSets�		&
	Specification and implementations of finite sets and finite maps (by lists and by AVL trees)
\\
Reals		&
	Axiomatization of real numbers (classical, basic functions, integer part, fractional part, limit, derivative, Cauchy series, power series and results,...)
\\
Relations		&
	Relations (definitions and basic results)
\\
Sorting		&
	Sorted list (basic definitions and heapsort correctness)
\\
Strings�		&
	8-bits characters and strings
\\
Wellfounded		&
	Well-founded relations (basic results)
\\
\end{tabular}




\subsection{Require} \label{require}
To use the standard libraries or other compiled files, you first need to tell the environment that it needs to load the compiled file. 
Require adds the specified module and all of its dependencies to the environment. 

\begin{code}
	\cmd{Require} Logic.
\end{code}

\noindent
Require Import loads the specified module and its dependences, then imports the contents of the specified module.

\begin{code}
	\cmd{Require Import} Bool.
\end{code}

\noindent
Require Export acts like Require Import, but will ensure that any module B that uses Require Import on the module A that contained the Require Export command will import both module A and the one specified in the Require Export command.

\begin{code}
	\cmd{Require Export} Bool.
\end{code}




\subsection{Load} \label{load}
This is used to load a file or library into the current environment. 
To load one of the standard libraries, you can simply use the Load command.

\begin{code}
	\cmd{Load} Arith.
\end{code}

\noindent
However, to load any other existing file, you will likely need to specify where to look for the file; to do this, there is the Add LoadPath command.
All the commands in the loaded file will be evaluated

\begin{code}
	\cmd{Add} LoadPath \str{``myDirectory/path"}. \\
	\cmd{Load} myFile. 
\end{code}








