
Please see file ``Defining.v" to follow along with this code in the CoqIDE. 

\subsection{Sorts} \label{subsec: sort}
There are three main Sorts for defining types: {\bf Prop}, {\bf Set}, and {\bf Type}. 
\begin{itemize}
	\item {\bf Prop}: This type is for logical propositions. 
	\item {\bf Set}: This type is for small sets, such as booleans (bool) and natural numbers (nat). 
	\item {\bf Type}: This type can be used for small sets as well as larger sets - it encompasses both {\bf Prop} and {\bf Set}. 
\end{itemize}

\noindent
These are used in defining Inductive types, as shown in the next section.




\subsection{Inductive} \label{subsec: inductive}
The command Inductive is used to define simple inductive types and the constructors used in the type. 
The name is placed directly after the keyword Inductive, when a colon followed by the Sort 
(discussed in the previous section) of the inductive type you are defining. 
This is followed by $:=$ and the constructors (i.e. elements) included in that type.  
For example, boolean values are defined in Coq as follows:

\hspace{-1cm}
\begin{tabular}{p{8cm} p{8cm}}
	\begin{code}
		\df{Inductive} \nm{bool} : \ty{Set} := 	\\ \-\ \quad
 		 $\mid$ true						\\ \-\ \quad
 		 $\mid$ false.						\\		
	\end{code}
&
	\begin{msg}
		bool is defined			\\
		bool\_rect is defined		\\
		bool\_ind is defined		\\
		bool\_rec is defined
	\end{msg}
\end{tabular}

\noindent
and nat numbers are defined as:

\hspace{-1cm}
\begin{tabular}{p{8cm} p{8cm}}
	\begin{code}
		\df{Inductive} \nm{nat} : \ty{Set} := 		\\ \-\ \quad
 		 $\mid$ O	: nat						\\ \-\ \quad
 		 $\mid$ S : nat $->$ nat.				\\		
	\end{code}
&
	\begin{msg}
		nat is defined			\\
		nat\_rect is defined		\\
		nat\_ind is defined		\\
		nat\_rec is defined
	\end{msg}
\end{tabular}

\noindent
You can see what all the definitions it created are by using Print:

\begin{code}	\cmd{Print} nat.			\end{code}
\begin{msg}
	\df{Inductive} nat : \ty{Set} :=  O : \nm{nat} $\mid$ S : \nm{nat} $->$ \nm{nat}
\end{msg}
\begin{code}	\cmd{Print} nat\_rect.		\end{code}
\begin{msg}
	nat\_rect = 	\\
	\kw{fun} (P : \nm{nat} $->$ \ty{Type}) (f : P \nm{O}) (f0 : \kw{forall} n : \nm{nat}, P n $->$ P (\nm{S} n)) $=>$	\\
	\kw{fix} F (n : \nm{nat}) : P n := \kw{match} n \kw{as} n0 \kw{return} (P n0) \kw{with}		\\ \-\ \qquad\qquad
			$\mid$ \nm{O} $=>$ f									\\ \-\ \qquad\qquad
			$\mid$ \nm{S} n0 $=>$ f0 n0 (F n0)							\\ \-\ \qquad\qquad
			\kw{end}												\\ \-\ \qquad
		: \kw{forall} P : \nm{nat} $->$ \ty{Type}, P \nm{O} $->$ (\kw{forall} n : \nm{nat}, P n 
			$->$ P (\nm{S} n)) $->$ \kw{forall} n : \nm{nat}, P n
	\\ \\
	Argument scopes are [function\_scope \_ function\_scope \_\ ]
\end{msg}
\begin{code}	\cmd{Print} nat\_ind.		\end{code}
\begin{msg}
	nat\_rect = 	\\
	\kw{fun} (P : \nm{nat} $->$ \ty{Prop}) (f : P \nm{O}) (f0 : \kw{forall} n : \nm{nat}, P n $->$ P (\nm{S} n)) $=>$	\\
	\kw{fix} F (n : \nm{nat}) : P n := \kw{match} n \kw{as} n0 \kw{return} (P n0) \kw{with}		\\ \-\ \qquad\qquad
			$\mid$ \nm{O} $=>$ f									\\ \-\ \qquad\qquad
			$\mid$ \nm{S} n0 $=>$ f0 n0 (F n0)							\\ \-\ \qquad\qquad
			\kw{end}												\\ \-\ \qquad
		: \kw{forall} P : \nm{nat} $->$ \ty{Prop}, P \nm{O} $->$ (\kw{forall} n : \nm{nat}, P n 
			$->$ P (\nm{S} n)) $->$ \kw{forall} n : \nm{nat}, P n
	\\ \\
	Argument scopes are [function\_scope \_ function\_scope \_\ ]
\end{msg}
\begin{code}	\cmd{Print} nat\_rec.		\end{code}
\begin{msg}	
	nat\_rec = 								\\
	\kw{fun} P : \nm{nat} $->$ \ty{Set} $=>$ \nm{nat\_rect} P				\\ \-\ \qquad
		: \kw{forall} P : \nm{nat} -$>$ \ty{Set},					
		  P \nm{O} $->$						
       			(\kw{forall} n : \nm{nat}, P n $->$ P (\nm{S} n)) $->$ \kw{forall} n : \nm{nat}, P n 
	\\ \\
	Argument scopes are [function\_scope \_ function\_scope \_\ ]
\end{msg}

\noindent
You can use these properties of what you've defined in proofs.

~\\
\noindent
Similarly, you can define days of the week:

\hspace{-1cm}
\begin{tabular}{p{8cm} p{8cm}}
\begin{code}
\Inductive \nm{day}: \Type :=	\\ \-\quad
 $\mid$ monday : day		\\ \-\quad
 $\mid$ tuesday : day		\\ \-\quad
 $\mid$ wednesday : day		\\ \-\quad
 $\mid$ thursday : day		\\ \-\quad
 $\mid$ friday : day			\\ \-\quad
 $\mid$ saturday : day		\\ \-\quad
 $\mid$ sunday : day.
\end{code}
&
\begin{msg}			
day is defined			\\
day\_rect is defined		\\
day\_ind is defined		\\
day\_rec is defined		
\end{msg}
\end{tabular}


\noindent
or lists of natural numbers:

\hspace{-1cm}
\begin{tabular}{p{8cm} p{8cm}}
\begin{code}
\Inductive \nm{natlist} \ty{Type} :=			\\ \-\ \quad
$\mid$ nil								\\ \-\ \quad
$\mid$ cons (n:Datatypes.nat) (l:natlist).		\\
\end{code}
&
\begin{msg}
list is defined			\\
list\_rect is defined		\\
list\_ind is defined		\\
list\_rec is defined		
\end{msg}
\end{tabular}


\noindent 
It is also possible to define polymorphic lists:

\hspace{-1cm}
\begin{tabular}{p{8cm} p{8cm}}
\begin{code}
\Inductive \nm{list} (A: \ty{Set}) : \ty{Set} :=	\\ \-\ \quad
$\mid$ nil : list A						\\ \-\ \quad
$\mid$ cons : A $->$ list A.				\\
\end{code}
&
\begin{msg}
list is defined			\\
list\_rect is defined		\\
list\_ind is defined		\\
list\_rec is defined		
\end{msg}
\end{tabular}

(polymorphic lists will not be used in the remainder of the tutorial 
- it is just an example of a more complex construct that can be defined in COQ). 






\subsection{Definition} \label{subsec: definition}

The command Definition is used to bind an name to some term. 
The name is always placed directly after the keyword Definition, 
and the term to bind to the name is given after $:=$. 
For example, we can give the name $x$ a simple value of 4: 

\hspace{-1cm}
\begin{tabular}{p{8cm} p{8cm}}
\begin{code}
\Definition \nm{x} := 4.
\end{code}
& 
\begin{msg}
x is defined
\end{msg}
\end{tabular}

\noindent
or we can use this to define functions, such as the following simple function 
(using the previous weekday inductive type definition), 
taking a weekday as input and giving back the weekday as output. 
Here, we are specifying the parameter $d$ of type $day$ must be given to the function. 
When giving a parameter, you give the $(paramName: type)$ as in $(d:day)$. 
The parameters are then followed by the return type, as in $: returnType$. 
This is shown in the following example. 

\hspace{-1cm}
\begin{tabular}{p{8cm} p{8cm}}
\begin{code}
\Definition \nm{next\_weekday} (d:day) : day :=	\\ \-\ \quad
  \match d \with							\\ \-\ \qquad
   $\mid$ monday $=>$ tuesday				\\ \-\ \qquad
   $\mid$ tuesday $=>$ wednesday			\\ \-\ \qquad
   $\mid$ wednesday $=>$ thursday			\\ \-\ \qquad
   $\mid$ thursday $=>$ friday				\\ \-\ \qquad
   $\mid$ friday $=>$ monday				\\ \-\ \qquad
   $\mid$ saturday $=>$ monday			\\ \-\ \qquad
   $\mid$ sunday $=>$ monday				\\ \-\ \quad
  \End.
\end{code}
& 
\begin{msg}
next\_weekday is defined
\end{msg}
\end{tabular}

\noindent
Another simple example function definition 
(using the previous nat number inductive type definition) 
is taking a nat number and returning that result of adding 2 to that nat number: 

\hspace{-1cm}
\begin{tabular}{p{8cm} p{8cm}}
\begin{code}
\Definition \nm{plus2} (n:nat) : nat :=		\\ \-\ \quad
  \match n \with						\\ \-\ \qquad
    $\mid$ O $=>$ S (S O)				\\ \-\ \qquad
    $\mid$ \_ $=>$ S (S n)				\\ \-\ \quad
  \End.
\end{code}
& 
\begin{msg}
plus2 is defined
\end{msg}
\end{tabular}

\noindent
You can also give multiple parameters, as shown in the example below using 3 parameters: 

\hspace{-1cm}
\begin{tabular}{p{12cm} p{4cm}}
\begin{code}
\Definition \nm{choose1} (b: bool) (n1: Datatypes.nat) (n2: Datatypes.nat) : Datatypes.nat :=	\\ \-\ \quad
  \match b \with										\\ \-\ \qquad
    $\mid$ true $=>$ n1								\\ \-\ \qquad
    $\mid$ false $=>$ n2								\\ \-\ \quad
  \End.
\end{code}
& 
\begin{msg}
choose1 is defined
\end{msg}
\end{tabular}

\noindent
We have to specify that we would like to use Datatypes.nat as our type in order to use 
natural numbers (i.e. 0, 1, 2, 3, 4, 5, 6, 7, 8, 9), as we have not specified how to interpret these 
numbers from our definition of nat (you can do this using the Notation command 
- using this command will be discussed in the following subsection \ref{subsec: notation}). 
Alternately, we can define the same function without giving the types of the parameters 
(however, it is always best practice to declare the types of all parameters, 
to ensure they are interpreted as you expect them to be). 
When you do not specify the types of parameters, the parentheses around the 
parameter names are optional. 

\hspace{-1cm}
\begin{tabular}{p{12cm} p{4cm}}
\begin{code}
\Definition \nm{choose1$'$} b n1 n2 : Datatypes.nat :=		\\ \-\ \quad
  \match b \with										\\ \-\ \qquad
    $\mid$ true $=>$ n1								\\ \-\ \qquad
    $\mid$ false $=>$ n2								\\ \-\ \quad
  \End.
\end{code}
& 
\begin{msg}
choose1$'$ is defined
\end{msg}
\end{tabular}

\noindent
Both {\it $choose$1} and {\it $choose$1$'$} have the same functionality. 






\subsection{Notation} \label{subsec: notation}

The command Notation is used to define a short-hand way of writing a concept we have previously defined 
- for example, take our list definition from subsection \ref{subsec: inductive}. 
We can then make the shorthand notation where $[\ ]$ refers to $nil$, or the empty list, and 
$x :: xs$ refers to $(cons\ x\ xs)$, or appending some number $x$ to some list $xs$. 

\hspace{-1cm}
\begin{tabular}{p{8cm} p{8cm}}
\begin{code}
\Notation \str{``[ ]''} := nil. 			\\
\Notation \str{``x :: xs''} := (cons x xs).
\end{code}
& 
\begin{msg}
Setting notation at level 0.		\\
\end{msg}
\end{tabular}


\noindent
Now we are able to write lists more simply (shown compared to the identical list using $cons$ and $nil$): 

\hspace{-1cm}
\begin{tabular}{p{8cm} p{8cm}}
\begin{code}
Compute 3::2::1::[ ].	\\
Compute (cons 3 (cons 2 (cons 1 nil))).
\end{code}
& 
\begin{msg}
= 3 :: 2 :: 1 :: [ ] \\
: natlist
\end{msg}
\end{tabular}

\noindent
This can be used to define any number of concepts that you may desire, 
including notations for the boolean expressions for checking where some number is greater than 
some other number (which are not natively defined, as discussed in subsection \ref{subsec: bool_expr}). 
These notations and definitions are shown in example \ref{bool_expr}. 






\subsection{Fixpoint} \label{subsec: fixpoint}






\subsection{Compute} \label{subsec: compute}

The command Compute evaluates the term that follows it. 








