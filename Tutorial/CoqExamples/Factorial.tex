
See file ``factorial.v'' to follow along. 

~\\
For the factorial function (i.e., \TT{fact n} or \TT{n!} for short), \TT{0!} should return \TT{1}, 
and all other positive numbers should return the result of the multiplication of the given number 
\TT{n} with all numbers from \TT{1, ..., n-1}. 
To write this recursively, we can say \TT{fact n = n * fact (n -1)}. 
The factorial function does not work with negative numbers 
-- therefore, we can use \TT{nat} numbers to ensure we are only able to use it on numbers \TT{$>= 0$}. 
Here, we first load the \TT{Arith} library to use \TT{nat} numbers and give the definition of factorial. 
Because Coq defines \TT{nat} numbers as being \TT{O $\mid$ S \_}, we can leverage pattern matching to 
get the following definition: 

\begin{code}
	\Load \nm{Arith}.
	\\ \\
	\Fixpoint \nm{fact} (n:nat) := 				\\ \-\ \quad
 		\match n \with 						\\ \-\ \qquad
  			$\mid$ O $=>$ 1				\\ \-\ \qquad
    			$\mid$ S m $=>$ n * fact(m)		\\ \-\ \quad
  		\End.
\end{code}

\noindent
Here, we are matching \TT{n} with either \TT{O} or \TT{S m}, where \TT{m = n-1}, 
which allows us to state that in the case of \TT{n == S m}, we need to perform \TT{n * fact(m)}. 








