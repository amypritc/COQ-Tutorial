
See Coq file ``rev\_list.v'' to follow along with this example. 


~\\
In this example, we want to prove that the reverse of the reverse of a list is the original list 
(i.e., \TT{rev (rev xs) = xs}). 
In order to do this, we must first prove a few supplementary lemmas in order to obtain 
our original goal. 
First, in order to use list notations, we must either define them, or use the command 
\TT{\Load List.}:

\hspace{-1cm}
\begin{tabular}{l}\begin{code}		\Load List.			\end{code}\end{tabular}

~\\
Then we start with defining \TT{list\_nil}, a lemma to show that a list \TT{xs} appended to 
nil (i.e., the empty list \TT{[ ]}) is \TT{xs}: 

\hspace{-1cm}
\begin{tabular}{p{8cm} p{8cm}}
\begin{code}
\Lemma \nm{list\_nil} : 			\\
\Forall (xs : Datatypes.list nat),		\\
xs ++ [ ] = xs.	
\end{code}
&
\begin{goal}
1 subgoal	\\
\_\_\_\_\_\_\_\_\_\_\_\_\_\_\_\_\_\_\_\_\_\_\_\_\_\_\_\_\_\_\_\_\_\_\_\_\_\_(1/1)	\\
\Forall xs : Datatypes.list nat, 		
xs ++ [\ ] = xs
\end{goal}
\end{tabular}

\noindent
Next, we open the proof environment using the command \TT{\Proof.}: 

\hspace{-1cm}
\begin{tabular}{p{8cm} p{8cm}}
\begin{code}
\Proof.
\end{code}
&
\begin{goal}
1 subgoal	\\
\_\_\_\_\_\_\_\_\_\_\_\_\_\_\_\_\_\_\_\_\_\_\_\_\_\_\_\_\_\_\_\_\_\_\_\_\_\_(1/1)	\\
\Forall xs : Datatypes.list nat, 		
xs ++ [\ ] = xs
\end{goal}
\end{tabular}


\noindent
Then we introduce the variable we using for our list, \TT{xs}, using \TT{intros xs.}:

\hspace{-1cm}
\begin{tabular}{p{8cm} p{8cm}}
\begin{code}
intros xs.
\end{code}
&
\begin{goal}
1 subgoal														\\
xs : Datatypes.list nat											\\
\_\_\_\_\_\_\_\_\_\_\_\_\_\_\_\_\_\_\_\_\_\_\_\_\_\_\_\_\_\_\_\_\_\_\_\_\_\_(1/1)	\\
xs ++ [\ ] = xs
\end{goal}
\end{tabular}

\noindent
Next, we can use induction on the variable \TT{xs} to leverage the inductive nature of the list type 
and split our main goal into two subgoals: 

\hspace{-1cm}
\begin{tabular}{p{8cm} p{8cm}}
\begin{code}
induction xs.
\end{code}
&
\begin{goal}
2 subgoals													\\
\_\_\_\_\_\_\_\_\_\_\_\_\_\_\_\_\_\_\_\_\_\_\_\_\_\_\_\_\_\_\_\_\_\_\_\_\_\_(1/2)	\\
$[\ ] ++ [\ ] = [\ ]$												\\
\_\_\_\_\_\_\_\_\_\_\_\_\_\_\_\_\_\_\_\_\_\_\_\_\_\_\_\_\_\_\_\_\_\_\_\_\_\_(2/2)	\\
(a :: xs) $++$ [\ ] = a :: xs
\end{goal}
\end{tabular}

\noindent
Now, we enter into the first subgoal using \TT{-}:

\hspace{-1cm}
\begin{tabular}{p{8cm} p{8cm}}
\begin{code}
- 
\end{code}
&
\begin{goal}
1 subgoal														\\
\_\_\_\_\_\_\_\_\_\_\_\_\_\_\_\_\_\_\_\_\_\_\_\_\_\_\_\_\_\_\_\_\_\_\_\_\_\_(1/1)	\\
$[\ ] ++ [\ ] = [\ ]$												\\
															\\
Unfocused Goals:												\\
\_\_\_\_\_\_\_\_\_\_\_\_\_\_\_\_\_\_\_\_\_\_\_\_\_\_\_\_\_\_\_\_\_\_\_\_\_\_		\\
(a :: xs) ++ [\ ] = a :: xs											\\
\end{goal}
\end{tabular}

\noindent
We can use the tactic \TT{auto} to solve this subgoal:

\hspace{-1cm}
\begin{tabular}{p{8cm} p{8cm}}
\begin{code}
- auto. 
\end{code}
&
\begin{goal}
This subproof is complete, but there are some unfocused goals:			\\
\_\_\_\_\_\_\_\_\_\_\_\_\_\_\_\_\_\_\_\_\_\_\_\_\_\_\_\_\_\_\_\_\_\_\_\_\_\_		\\
(a :: xs) ++ [\ ] = a :: xs									
\end{goal}
\end{tabular}


\noindent
Now, we enter into the second subgoal using \TT{-}:

\hspace{-1cm}
\begin{tabular}{p{8cm} p{8cm}}
\begin{code}
- 
\end{code}
&
\begin{goal}
1 subgoal														\\
a : nat														\\
xs : Datatypes.list nat											\\
IHxs : xs ++ [\ ] = xs												\\
\_\_\_\_\_\_\_\_\_\_\_\_\_\_\_\_\_\_\_\_\_\_\_\_\_\_\_\_\_\_\_\_\_\_\_\_\_\_(1/1)	\\
(a :: xs) ++ [\ ] = a :: xs	
\end{goal}
\end{tabular}

\noindent
For this subgoal, we want to be able to use the inductive hypothesis, and to do that 
we can use the tactic \TT{simpl} to remove the parentheses around \TT{a :: xs}: 

\hspace{-1cm}
\begin{tabular}{p{8cm} p{8cm}}
\begin{code}
- simpl.
\end{code}
&
\begin{goal}
1 subgoal														\\
a : nat														\\
xs : Datatypes.list nat											\\
IHxs : xs ++ [\ ] = xs												\\
\_\_\_\_\_\_\_\_\_\_\_\_\_\_\_\_\_\_\_\_\_\_\_\_\_\_\_\_\_\_\_\_\_\_\_\_\_\_(1/1)	\\
a :: xs ++ [\ ] = a :: xs	
\end{goal}
\end{tabular}

\noindent
Now we can use the tactic \TT{rewrite ->} to use the inductive hypothesis to rewrite 
\TT{xs ++ [\ ]} to \TT{xs} on the left-hand side: 

\hspace{-1cm}
\begin{tabular}{p{8cm} p{8cm}}
\begin{code}
rewrite $->$ IHxs.
\end{code}
&
\begin{goal}
1 subgoal														\\
a : nat														\\
xs : Datatypes.list nat											\\
IHxs : xs ++ [\ ] = xs												\\
\_\_\_\_\_\_\_\_\_\_\_\_\_\_\_\_\_\_\_\_\_\_\_\_\_\_\_\_\_\_\_\_\_\_\_\_\_\_(1/1)	\\
a :: xs = a :: xs	
\end{goal}
\end{tabular}

\noindent
We can then use the tactic \TT{auto} to finish this subgoal: 

\hspace{-1cm}
\begin{tabular}{p{8cm} p{8cm}}
\begin{code} 
auto. 
\end{code}
&
\begin{goal}
No more subgoals.
\end{goal}
\end{tabular}

\noindent
Finally, we finish up the proof of \TT{list\_nil} with the proof completion command, \TT{\Qed.}: 

\hspace{-1cm}
\begin{tabular}{p{8cm} p{8cm}}
\begin{code} 
\Qed. 
\end{code}
&
\begin{msg}
list\_nil is defined
\end{msg}
\end{tabular}

\noindent
Next, we need to define a lemma for list associativity, \TT{list\_assoc}:

\hspace{-1cm}
\begin{tabular}{p{7cm} p{9cm}}
\begin{code} 
\Lemma \nm{list\_assoc} : 				\\
\Forall (xs ys zs : Datatypes.list nat),			\\
(xs ++ ys) ++ zs = xs ++ (ys ++ zs).
\end{code}
&
\begin{goal}
1 subgoal														\\
\_\_\_\_\_\_\_\_\_\_\_\_\_\_\_\_\_\_\_\_\_\_\_\_\_\_\_\_\_\_\_\_\_\_\_\_\_\_(1/1)	\\
forall xs ys zs : Datatypes.list nat,									\\
(xs ++ ys) ++ zs = xs ++ ys ++ zs
\end{goal}
\end{tabular}

\noindent
Then we enter the proof environment, and introduce our local variables:

\hspace{-1cm}
\begin{tabular}{p{7cm} p{9cm}}
\begin{code} 
\Proof. 			\\
intros xs ys zs. 	
\end{code}
&
\begin{goal}
1 subgoal														\\
xs, ys, zs : Datatypes.list nat										\\
\_\_\_\_\_\_\_\_\_\_\_\_\_\_\_\_\_\_\_\_\_\_\_\_\_\_\_\_\_\_\_\_\_\_\_\_\_\_(1/1)	\\
(xs ++ ys) ++ zs = xs ++ ys ++ zs
\end{goal}
\end{tabular}

\noindent
We then specify we would like to complete the proof by using induction on list \TT{xs}:

\hspace{-1cm}
\begin{tabular}{p{7cm} p{9cm}}
\begin{code} 
induction xs.
\end{code}
&
\begin{goal}
2 subgoals
ys, zs : Datatypes.list nat											\\
\_\_\_\_\_\_\_\_\_\_\_\_\_\_\_\_\_\_\_\_\_\_\_\_\_\_\_\_\_\_\_\_\_\_\_\_\_\_(1/2)	\\
([\ ] ++ ys) ++ zs = [\ ] ++ ys ++ zs									\\
\_\_\_\_\_\_\_\_\_\_\_\_\_\_\_\_\_\_\_\_\_\_\_\_\_\_\_\_\_\_\_\_\_\_\_\_\_\_(2/2)	\\
((a :: xs) ++ ys) ++ zs = (a :: xs) ++ ys ++ zs
\end{goal}
\end{tabular}

\noindent
Then we can enter into the first subgoal, and complete the proof of this subgoal by simply 
using the tactic \TT{auto}:

\hspace{-1cm}
\begin{tabular}{p{7cm} p{9cm}}
\begin{code} 
- auto.
\end{code}
&
\begin{goal}
This subproof is complete, but there are some unfocused goals:			\\
\_\_\_\_\_\_\_\_\_\_\_\_\_\_\_\_\_\_\_\_\_\_\_\_\_\_\_\_\_\_\_\_\_\_\_\_\_\_(1/1)	\\
((a :: xs) ++ ys) ++ zs = (a :: xs) ++ ys ++ zs
\end{goal}
\end{tabular}

\noindent
Then we can enter into the second subgoal, and view our inductive hypothesis:

\hspace{-1cm}
\begin{tabular}{p{7cm} p{9cm}}
\begin{code} 
- 
\end{code}
&
\begin{goal}
1 subgoal														\\
a : nat														\\
xs, ys, zs : Datatypes.list nat										\\
IHxs : (xs ++ ys) ++ zs =   xs ++ ys ++ zs								\\
\_\_\_\_\_\_\_\_\_\_\_\_\_\_\_\_\_\_\_\_\_\_\_\_\_\_\_\_\_\_\_\_\_\_\_\_\_\_(1/1)	\\
((a :: xs) ++ ys) ++ zs = (a :: xs) ++ ys ++ zs
\end{goal}
\end{tabular}

\noindent
We can use the tactic \TT{simpl} to simplify our subgoal, removing some of the layers of parentheses:

\hspace{-1cm}
\begin{tabular}{p{7cm} p{9cm}}
\begin{code} 
- simpl.
\end{code}
&
\begin{goal}
1 subgoal														\\
a : nat														\\
xs, ys, zs : Datatypes.list nat										\\
IHxs : (xs ++ ys) ++ zs =   xs ++ ys ++ zs								\\
\_\_\_\_\_\_\_\_\_\_\_\_\_\_\_\_\_\_\_\_\_\_\_\_\_\_\_\_\_\_\_\_\_\_\_\_\_\_(1/1)	\\
a :: (xs ++ ys) ++ zs = a :: xs ++ ys ++ zs
\end{goal}
\end{tabular}

\noindent
Now we can use the \TT{rewrite} tactic to further our proof by using the inductive hypothesis. 
As you can see above, in the right-hand side of our subgoal, 
we have the term sequence matching that of the right-hand side of the inductive hypothesis, 
and likewise with on left-hand sides of the subgoal and inductive hypothesis. 
In the last proof, we used \TT{->}, so this time we will use \TT{<-} with \TT{rewrite} to 
illustrate the difference:

\hspace{-1cm}
\begin{tabular}{p{7cm} p{9cm}}
\begin{code} 
rewrite $<-$ IHxs.
\end{code}
&
\begin{goal}
1 subgoal														\\
a : nat														\\
xs, ys, zs : Datatypes.list nat										\\
IHxs : (xs ++ ys) ++ zs = xs ++ ys ++ zs								\\
\_\_\_\_\_\_\_\_\_\_\_\_\_\_\_\_\_\_\_\_\_\_\_\_\_\_\_\_\_\_\_\_\_\_\_\_\_\_(1/1)	\\
a :: (xs ++ ys) ++ zs = a :: (xs ++ ys) ++ zs
\end{goal}
\end{tabular}

\noindent
Now we are left with identical left- and right-hand sides of the subgoal. 
From here, we can finish the proof of this subgoal using the tactic \TT{auto}:

\hspace{-1cm}
\begin{tabular}{p{7cm} p{9cm}}
\begin{code} 
auto.
\end{code}
&
\begin{goal}
No more subgoals.
\end{goal}
\end{tabular}

\noindent
Finally, we finish up the proof of \TT{list\_assoc} with the proof completion command, \TT{\Qed.}: 

\hspace{-1cm}
\begin{tabular}{p{7cm} p{9cm}}
\begin{code} 
\Qed. 
\end{code}
&
\begin{msg}
list\_assoc is defined
\end{msg}
\end{tabular}

\noindent
We now progress on to the lemma (\TT{rev\_list}) for proving that the reverse of two lists appended together 
(i.e., \TT{xs} appended to \TT{ys}) is equivalent to the reverse of the second list (\TT{ys}) 
appended to the reverse of the first list (\TT{xs}).

\hspace{-1cm}
\begin{tabular}{p{7cm} p{9cm}}
\begin{code} 
\Lemma \nm{rev\_list} : 				\\
\Forall (xs ys : Datatypes.list nat),		\\
rev (xs ++ ys) = (rev ys) ++ (rev xs).
\end{code}
&
\begin{goal}
1 subgoal														\\
\_\_\_\_\_\_\_\_\_\_\_\_\_\_\_\_\_\_\_\_\_\_\_\_\_\_\_\_\_\_\_\_\_\_\_\_\_\_(1/1)	\\
forall xs ys : Datatypes.list nat,										\\
rev (xs ++ ys) = rev ys ++ rev xs					
\end{goal}
\end{tabular}

\noindent
As before, we enter the proof environment, and introduce our local variables:

\hspace{-1cm}
\begin{tabular}{p{7cm} p{9cm}}
\begin{code} 
\Proof. 			\\
intros xs ys. 	
\end{code}
&
\begin{goal}
1 subgoal														\\
xs, ys : Datatypes.list nat											\\
\_\_\_\_\_\_\_\_\_\_\_\_\_\_\_\_\_\_\_\_\_\_\_\_\_\_\_\_\_\_\_\_\_\_\_\_\_\_(1/1)	\\
rev (xs ++ ys) = rev ys ++ rev xs
\end{goal}
\end{tabular}

\noindent
Then we declare we want to solve the proof by induction over \TT{xs}:

\hspace{-1cm}
\begin{tabular}{p{7cm} p{9cm}}
\begin{code} 
induction xs. 	
\end{code}
&
\begin{goal}
2 subgoals													\\
ys : Datatypes.list nat											\\
\_\_\_\_\_\_\_\_\_\_\_\_\_\_\_\_\_\_\_\_\_\_\_\_\_\_\_\_\_\_\_\_\_\_\_\_\_\_(1/2)	\\
rev ([\ ] ++ ys) = rev ys ++ rev [\ ]									\\
\_\_\_\_\_\_\_\_\_\_\_\_\_\_\_\_\_\_\_\_\_\_\_\_\_\_\_\_\_\_\_\_\_\_\_\_\_\_(2/2)	\\
rev ((a :: xs) ++ ys) = rev ys ++ rev (a :: xs)
\end{goal}
\end{tabular}

\noindent
Next, we enter into the first subgoal, an we can use \TT{simpl} to simplify the subgoal:

\hspace{-1cm}
\begin{tabular}{p{7cm} p{9cm}}
\begin{code} 
- simpl. 	
\end{code}
&
\begin{goal}
1 subgoal														\\
ys : Datatypes.list nat											\\
\_\_\_\_\_\_\_\_\_\_\_\_\_\_\_\_\_\_\_\_\_\_\_\_\_\_\_\_\_\_\_\_\_\_\_\_\_\_(1/1)	\\
rev ys = rev ys ++ [\ ]												\\
															\\
Unfocused Goals:												\\
\_\_\_\_\_\_\_\_\_\_\_\_\_\_\_\_\_\_\_\_\_\_\_\_\_\_\_\_\_\_\_\_\_\_\_\_\_\_		\\
rev ((a :: xs) ++ ys) = rev ys ++ rev (a :: xs)
\end{goal}
\end{tabular}

\noindent
We are now able to use the \TT{rewrite} tactic with our first lemma \TT{list\_nil} to further the proof:

\hspace{-1cm}
\begin{tabular}{p{7cm} p{9cm}}
\begin{code} 
rewrite $->$ list\_nil. 	
\end{code}
&
\begin{goal}
1 subgoal														\\
ys : Datatypes.list nat											\\
\_\_\_\_\_\_\_\_\_\_\_\_\_\_\_\_\_\_\_\_\_\_\_\_\_\_\_\_\_\_\_\_\_\_\_\_\_\_(1/1)	\\
rev ys = rev ys 													\\
															\\
Unfocused Goals:												\\
\_\_\_\_\_\_\_\_\_\_\_\_\_\_\_\_\_\_\_\_\_\_\_\_\_\_\_\_\_\_\_\_\_\_\_\_\_\_		\\
rev ((a :: xs) ++ ys) = rev ys ++ rev (a :: xs)
\end{goal}
\end{tabular}

\noindent
Since both sides of our subgoal equation are equal, we can use the tactic \TT{reflexivity} 
to complete this subgoal:

\hspace{-1cm}
\begin{tabular}{p{7cm} p{9cm}}
\begin{code} 
reflexivity. 	
\end{code}
&
\begin{goal}
This subproof is complete, but there are some unfocused goals:			\\
\_\_\_\_\_\_\_\_\_\_\_\_\_\_\_\_\_\_\_\_\_\_\_\_\_\_\_\_\_\_\_\_\_\_\_\_\_\_(1/1)	\\
rev ((a :: xs) ++ ys) = rev ys ++ rev (a :: xs)
\end{goal}
\end{tabular}

\noindent
We then enter the second subgoal, and attempt to use the tactic \TT{simpl} to 
obtain a set of terms in the subgoal that will match the inductive hypothesis: 

\hspace{-1cm}
\begin{tabular}{p{7cm} p{9cm}}
\begin{code} 
- simpl. 	
\end{code}
&
\begin{goal}
1 subgoal														\\
a : nat														\\
xs, ys : Datatypes.list nat											\\
IHxs : rev (xs ++ ys) =  rev ys ++ rev xs								\\
\_\_\_\_\_\_\_\_\_\_\_\_\_\_\_\_\_\_\_\_\_\_\_\_\_\_\_\_\_\_\_\_\_\_\_\_\_\_(1/1)	\\
rev (xs ++ ys) ++ [a] = rev ys ++ rev xs ++ [a]
\end{goal}
\end{tabular}

\noindent
The use of tactic \TT{simpl} succeeded in obtaining \TT{rev (xs ++ ys)} as a subset 
of the terms on the left-hand side of the equation, so we can now use the tactic 
\TT{rewrite} with the inductive hypothesis to convert this subset of terms to 
\TT{ev ys ++ rev xs}: 

\hspace{-1cm}
\begin{tabular}{p{6.5cm} p{9.5cm}}
\begin{code} 
rewrite $->$ IHxs. 	
\end{code}
&
\begin{goal}
1 subgoal														\\
a : nat														\\
xs, ys : Datatypes.list nat											\\
IHxs : rev (xs ++ ys) =  rev ys ++ rev xs								\\
\_\_\_\_\_\_\_\_\_\_\_\_\_\_\_\_\_\_\_\_\_\_\_\_\_\_\_\_\_\_\_\_\_\_\_\_\_\_(1/1)	\\
(rev ys ++ rev xs) ++ [a] = rev ys ++ rev xs ++ [a]
\end{goal}
\end{tabular}

\noindent 
We can now use the tactic \TT{apply} to apply the \TT{list\_assoc} lemma we defined 
previously on our currently subgoal: 

\hspace{-1cm}
\begin{tabular}{p{6.5cm} p{9.5cm}}
\begin{code} 
apply list\_assoc. 	
\end{code}
&
\begin{goal}
No more subgoals.
\end{goal}
\end{tabular}

\noindent
Since we have proved both of our subgoals, we can now close out the proof with 
the command \TT{\Qed.}:

\hspace{-1cm}
\begin{tabular}{p{7cm} p{9cm}}
\begin{code} 
\Qed. 	
\end{code}
&
\begin{msg}
rev\_list is defined
\end{msg}
\end{tabular}

\noindent
Finally, we can come back to our main lemma that the reverse of the reverse of a list 
is itself, \TT{rev\_rev}: 

\hspace{-1cm}
\begin{tabular}{p{7cm} p{9cm}}
\begin{code} 
\Lemma \nm{rev\_rev} : 			\\
\Forall (xs : Datatypes.list nat),	\\
rev (rev xs) = xs.		
\end{code}
&
\begin{goal}
1 subgoal														\\
\_\_\_\_\_\_\_\_\_\_\_\_\_\_\_\_\_\_\_\_\_\_\_\_\_\_\_\_\_\_\_\_\_\_\_\_\_\_(1/1)	\\
\Forall xs : Datatypes.list nat,										\\
rev (rev xs) = xs
\end{goal}
\end{tabular}

\noindent
As before, we enter the proof environment, and introduce our local variable:

\hspace{-1cm}
\begin{tabular}{p{7cm} p{9cm}}
\begin{code} 
\Proof. 			\\
intros xs. 	
\end{code}
&
\begin{goal}
1 subgoal														\\
xs : Datatypes.list nat											\\
\_\_\_\_\_\_\_\_\_\_\_\_\_\_\_\_\_\_\_\_\_\_\_\_\_\_\_\_\_\_\_\_\_\_\_\_\_\_(1/1)	\\
rev (rev xs) = xs
\end{goal}
\end{tabular}

\noindent
Here we will again use induction on list \TT{xs}:

\hspace{-1cm}
\begin{tabular}{p{7cm} p{9cm}}
\begin{code} 
induction xs. 
\end{code}
&
\begin{goal}
2 subgoals													\\
\_\_\_\_\_\_\_\_\_\_\_\_\_\_\_\_\_\_\_\_\_\_\_\_\_\_\_\_\_\_\_\_\_\_\_\_\_\_(1/2)	\\
rev (rev [\ ]) = [\ ]												\\
\_\_\_\_\_\_\_\_\_\_\_\_\_\_\_\_\_\_\_\_\_\_\_\_\_\_\_\_\_\_\_\_\_\_\_\_\_\_(2/2)	\\
rev (rev (a :: xs)) = a :: xs
\end{goal}
\end{tabular}

\noindent
We enter the first subgoal, and attempt to use the tactic \TT{auto} to complete the proof:

\hspace{-1cm}
\begin{tabular}{p{7cm} p{9cm}}
\begin{code} 
- auto. 
\end{code}
&
\begin{goal}
This subproof is complete, but there are some unfocused goals:			\\
\_\_\_\_\_\_\_\_\_\_\_\_\_\_\_\_\_\_\_\_\_\_\_\_\_\_\_\_\_\_\_\_\_\_\_\_\_\_(1/1)	\\
rev (rev (a :: xs)) = a :: xs
\end{goal}
\end{tabular}

\noindent
Since \TT{auto} has completed this subgoal, we now move to the second subgoal. 
We first use the tactic \TT{simpl} to try to obtain either the inductive hypothesis, 
or a subset of terms as in one of the lemmas we have previously defined: 

\hspace{-1cm}
\begin{tabular}{p{7cm} p{9cm}}
\begin{code} 
- simpl. 
\end{code}
&
\begin{goal}
1 subgoal														\\
a : nat														\\
xs : Datatypes.list nat											\\
IHxs : rev (rev xs) = xs											\\
\_\_\_\_\_\_\_\_\_\_\_\_\_\_\_\_\_\_\_\_\_\_\_\_\_\_\_\_\_\_\_\_\_\_\_\_\_\_(1/1)	\\
rev (rev xs ++ [a]) = a :: xs
\end{goal}
\end{tabular}

\noindent
This has left us at a point where we have the reverse of two lists appended to 
each other (i.e., \TT{rev xs} and \TT{[a]}), so we can now use the tactic 
\TT{rewrite} with out previous lemma \TT{rev\_list}:  

\hspace{-1cm}
\begin{tabular}{p{7cm} p{9cm}}
\begin{code} 
rewrite $->$ rev\_list. 
\end{code}
&
\begin{goal}
1 subgoal														\\
a : nat														\\
xs : Datatypes.list nat											\\
IHxs : rev (rev xs) = xs											\\
\_\_\_\_\_\_\_\_\_\_\_\_\_\_\_\_\_\_\_\_\_\_\_\_\_\_\_\_\_\_\_\_\_\_\_\_\_\_(1/1)	\\
rev [a] ++ rev (rev xs) = a :: xs
\end{goal}
\end{tabular}

\noindent
Now we can see that we have the left-hand side of the inductive hypothesis as 
sub-terms in the left-hand side of the subgoal, so we can now use the tactic 
\TT{rewrite} with the inductive hypothesis: 

\hspace{-1cm}
\begin{tabular}{p{7cm} p{9cm}}
\begin{code} 
rewrite $->$ IHxs. 
\end{code}
&
\begin{goal}
1 subgoal														\\
a : nat														\\
xs : Datatypes.list nat											\\
IHxs : rev (rev xs) = xs											\\
\_\_\_\_\_\_\_\_\_\_\_\_\_\_\_\_\_\_\_\_\_\_\_\_\_\_\_\_\_\_\_\_\_\_\_\_\_\_(1/1)	\\
rev [a] ++ xs = a :: xs
\end{goal}
\end{tabular}

\noindent
We will now attempt to complete the proof of this subgoal using the tactic \TT{auto}: 

\hspace{-1cm}
\begin{tabular}{p{7cm} p{9cm}}
\begin{code} 
auto. 
\end{code}
&
\begin{goal}
No more subgoals.
\end{goal}
\end{tabular}

\noindent 
Since this has succeeded in proving our final subgoal, we can now finish our proof:

\hspace{-1cm}
\begin{tabular}{p{7cm} p{9cm}}
\begin{code} 
\Qed. 
\end{code}
&
\begin{msg}
rev\_rev is defined
\end{msg}
\end{tabular}




