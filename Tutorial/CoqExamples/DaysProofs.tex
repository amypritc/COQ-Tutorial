
Please see file ``days.v'' to follow along with this example. 

~\\
This example is fairly simple, giving some basic proofs Coq. 
Please see Example \ref{days} for the more on the definitions these proofs rely on. 

\begin{code}
\Definition \nm{next\_weekday} (d:day): day :=		\\ \-\ \quad
  \match d \with								\\ \-\ \qquad
   $\mid$ monday $=>$ tuesday					\\ \-\ \qquad
   $\mid$ tuesday $=>$ wednesday				\\ \-\ \qquad
   $\mid$ wednesday $=>$ thursday				\\ \-\ \qquad
   $\mid$ thursday $=>$ friday					\\ \-\ \qquad
   $\mid$ friday $=>$ monday					\\ \-\ \qquad
   $\mid$ saturday $=>$ monday				\\ \-\ \qquad
   $\mid$ sunday $=>$ monday					\\ \-\ \quad
  \End.
\end{code}

% PROOF 1
\noindent
We can do a simple proof to check that the results are as expected.

\hspace{-1cm}
\begin{tabular}{p{8cm} p{8cm}}
\begin{code}
	\Example \nm{test\_next\_weekday}:			\\ \-\ \quad
	  (next\_weekday (next\_weekday saturday)) 	\\ \-\ \qquad
	  	= tuesday.							\\
	\Proof.
\end{code}
&
\begin{goal}
1 subgoal			\\
\_\_\_\_\_\_\_\_\_\_\_\_\_\_\_\_\_\_\_\_\_\_\_\_\_\_\_\_\_\_\_\_\_\_\_\_\_\_(1/1)	\\
next\_weekday (next\_weekday saturday) = tuesday
\end{goal}
\end{tabular}

\noindent 
First, we can apply tactic \TT{simpl} to evaluate the uses of the \TT{next\_weekday} function: 

\hspace{-1cm}
\begin{tabular}{p{8cm} p{8cm}}
\begin{code}
simpl.
\end{code}
&
\begin{goal}
1 subgoal			\\
\_\_\_\_\_\_\_\_\_\_\_\_\_\_\_\_\_\_\_\_\_\_\_\_\_\_\_\_\_\_\_\_\_\_\_\_\_\_(1/1)	\\
tuesday = tuesday
\end{goal}
\end{tabular}

\noindent 
We can then use the tactic \TT{reflexivity} to show that \TT{tuesday} is equal to \TT{tuesday}: 

\hspace{-1cm}
\begin{tabular}{p{8cm} p{8cm}}
\begin{code}
reflexivity.
\end{code}
&
\begin{goal}
No more subgoals.
\end{goal}
\end{tabular}

\noindent 
Finally, we finish our proof with the command \TT{\Qed}:

\hspace{-1cm}
\begin{tabular}{p{8cm} p{8cm}}
\begin{code}
\Qed. 
\end{code}
&
\begin{msg}
test\_next\_weekday is defined
\end{msg}
\end{tabular}



% PROOF 2
\noindent 
We can also do a simple proof to show something will not be true, like that 
\TT{next\_weekday monday} is not \TT{friday} ($<>$ is used to say that two terms are not equal): 

\hspace{-1cm}
\begin{tabular}{p{8cm} p{8cm}}
\begin{code}
\Example \nm{next\_weekday\_monday\_friday}: \\ \-\ \quad 
  next\_weekday monday $<>$ friday. 		\\
\Proof. 
\end{code}
&
\begin{msg}
1 subgoal				\\
\_\_\_\_\_\_\_\_\_\_\_\_\_\_\_\_\_\_\_\_\_\_\_\_\_\_\_\_\_\_\_\_\_\_\_\_\_\_(1/1)	\\
next\_weekday monday $<>$ friday		
\end{msg}
\end{tabular}

\noindent 
This proof can be solved with a single tactic, \TT{discriminate}, which can be used to prove that two elements of an inductive type are different from each other: 

\hspace{-1cm}
\begin{tabular}{p{8cm} p{8cm}}
\begin{code}
discriminate. 
\end{code}
&
\begin{msg}
No more subgoals.
\end{msg}
\end{tabular}

\noindent 
Again, we finish our proof with the command \TT{\Qed}:

\hspace{-1cm}
\begin{tabular}{p{8cm} p{8cm}}
\begin{code}
\Qed. 
\end{code}
&
\begin{msg}
next\_weekday\_monday\_friday is defined
\end{msg}
\end{tabular}



% PROOF 3
\noindent
We can also prove more general things, such as that \TT{sunday} is not the next weekday for any given day \TT{d}: 

\hspace{-1cm}
\begin{tabular}{p{8cm} p{8cm}}
\begin{code}
\Example \nm{next\_weekday\_is\_not\_sunday}: 		\\
\Forall (d:day), next\_weekday d $<>$ sunday.		\\
\Proof. 
\end{code}
&
\begin{goal}
1 subgoal				\\
\_\_\_\_\_\_\_\_\_\_\_\_\_\_\_\_\_\_\_\_\_\_\_\_\_\_\_\_\_\_\_\_\_\_\_\_\_\_(1/1)	\\
\Forall d : day, next\_weekday d $<>$ sunday
\end{goal}
\end{tabular}

\noindent
Since we are using the variable \TT{d}, we need to introduce it into the proof environment: 

\hspace{-1cm}
\begin{tabular}{p{8cm} p{8cm}}
\begin{code}
intros d. 
\end{code}
&
\begin{goal}
1 subgoal				\\
d : day				\\
\_\_\_\_\_\_\_\_\_\_\_\_\_\_\_\_\_\_\_\_\_\_\_\_\_\_\_\_\_\_\_\_\_\_\_\_\_\_(1/1)	\\
next\_weekday d $<>$ sunday
\end{goal}
\end{tabular}

\noindent
Here, we can use the \TT{destruct} tactic to do case analysis over all possible days that \TT{d} can be, giving us 7 subgoals (i.e., one for each possible day of the week): 

\hspace{-1cm}
\begin{tabular}{p{8cm} p{8cm}}
\begin{code}
destruct d.
\end{code}
&
\begin{goal}		
7 subgoals													\\
\_\_\_\_\_\_\_\_\_\_\_\_\_\_\_\_\_\_\_\_\_\_\_\_\_\_\_\_\_\_\_\_\_\_\_\_\_\_(1/7)	\\
next\_weekday monday $<>$ sunday								\\
\_\_\_\_\_\_\_\_\_\_\_\_\_\_\_\_\_\_\_\_\_\_\_\_\_\_\_\_\_\_\_\_\_\_\_\_\_\_(2/7)	\\
next\_weekday tuesday $<>$ sunday								\\
\_\_\_\_\_\_\_\_\_\_\_\_\_\_\_\_\_\_\_\_\_\_\_\_\_\_\_\_\_\_\_\_\_\_\_\_\_\_(3/7)	\\
next\_weekday wednesday $<>$ sunday								\\
\_\_\_\_\_\_\_\_\_\_\_\_\_\_\_\_\_\_\_\_\_\_\_\_\_\_\_\_\_\_\_\_\_\_\_\_\_\_(4/7)	\\
next\_weekday thursday $<>$ sunday								\\	
\_\_\_\_\_\_\_\_\_\_\_\_\_\_\_\_\_\_\_\_\_\_\_\_\_\_\_\_\_\_\_\_\_\_\_\_\_\_(5/7)	\\
next\_weekday friday $<>$ sunday									\\
\_\_\_\_\_\_\_\_\_\_\_\_\_\_\_\_\_\_\_\_\_\_\_\_\_\_\_\_\_\_\_\_\_\_\_\_\_\_(6/7)	\\
next\_weekday saturday $<>$ sunday								\\
\_\_\_\_\_\_\_\_\_\_\_\_\_\_\_\_\_\_\_\_\_\_\_\_\_\_\_\_\_\_\_\_\_\_\_\_\_\_(7/7)	\\
next\_weekday sunday $<>$ sunday
\end{goal}
\end{tabular}

\noindent
We now need to go through each subgoal and prove that it is not equal. 
Like in the last proof we did, we will use the tactic \TT{discriminate} 
-- we can first apply \TT{simpl} to show the result of the given evaluation of \TT{next\_weekday}. 
For example, we enter into the first subgoal:

\hspace{-1cm}
\begin{tabular}{p{8cm} p{8cm}}
\begin{code}
- 
\end{code}
&
\begin{goal}
1 subgoal														\\
\_\_\_\_\_\_\_\_\_\_\_\_\_\_\_\_\_\_\_\_\_\_\_\_\_\_\_\_\_\_\_\_\_\_\_\_\_\_(1/1)	\\
next\_weekday monday $<>$ sunday						
\end{goal}
\end{tabular}

\noindent
Then we call \TT{simpl}:

\hspace{-1cm}
\begin{tabular}{p{8cm} p{8cm}}
\begin{code}
- simpl. 
\end{code}
&
\begin{goal}
1 subgoal														\\
\_\_\_\_\_\_\_\_\_\_\_\_\_\_\_\_\_\_\_\_\_\_\_\_\_\_\_\_\_\_\_\_\_\_\_\_\_\_(1/1)	\\
tuesday $<>$ sunday
\end{goal}
\end{tabular}

\noindent
Then we can use \TT{discriminate} to finish this subgoal:

\hspace{-1cm}
\begin{tabular}{p{8cm} p{8cm}}
\begin{code}
discriminate.
\end{code}
&
\begin{goal}
This subproof is complete, but there are some unfocused goals:	\\ 	\\
\_\_\_\_\_\_\_\_\_\_\_\_\_\_\_\_\_\_\_\_\_\_\_\_\_\_\_\_\_\_\_\_\_\_\_\_\_\_(1/6)	\\
next\_weekday tuesday $<>$ sunday								\\
\_\_\_\_\_\_\_\_\_\_\_\_\_\_\_\_\_\_\_\_\_\_\_\_\_\_\_\_\_\_\_\_\_\_\_\_\_\_(2/6)	\\
next\_weekday wednesday $<>$ sunday								\\
\_\_\_\_\_\_\_\_\_\_\_\_\_\_\_\_\_\_\_\_\_\_\_\_\_\_\_\_\_\_\_\_\_\_\_\_\_\_(3/6)	\\
next\_weekday thursday $<>$ sunday								\\	
\_\_\_\_\_\_\_\_\_\_\_\_\_\_\_\_\_\_\_\_\_\_\_\_\_\_\_\_\_\_\_\_\_\_\_\_\_\_(4/6)	\\
next\_weekday friday $<>$ sunday									\\
\_\_\_\_\_\_\_\_\_\_\_\_\_\_\_\_\_\_\_\_\_\_\_\_\_\_\_\_\_\_\_\_\_\_\_\_\_\_(5/6)	\\
next\_weekday saturday $<>$ sunday								\\
\_\_\_\_\_\_\_\_\_\_\_\_\_\_\_\_\_\_\_\_\_\_\_\_\_\_\_\_\_\_\_\_\_\_\_\_\_\_(6/6)	\\
next\_weekday sunday $<>$ sunday
\end{goal}
\end{tabular}

\noindent
The proofs for each of these subgoals can be simply solved using \TT{discriminate} 
(given they are nearly identical to the above, they are shown grouped together): 

\hspace{-1cm}
\begin{tabular}{p{8cm} p{8cm}}
\begin{code}
- discriminate.		\\
- discriminate.		\\
- discriminate.		\\
- discriminate.		\\
- discriminate.		\\
- discriminate.	
\end{code}
&
\begin{goal}
No more subgoals.
\end{goal}
\end{tabular}

\noindent
As always, we close out the completed proof using \TT{\Qed}:

\hspace{-1cm}
\begin{tabular}{p{8cm} p{8cm}}
\begin{code}
\Qed. 
\end{code}
&
\begin{msg}
next\_weekday\_is\_not\_sunday is defined
\end{msg}
\end{tabular}



% PROOF 4
\noindent
Now, you may be wondering why there isn't a simpler way to show that all cases can be solved using the same tactic. 
There is, in fact, syntax to do just that -- so we can prove the same thing regarding \TT{saturday} as we did above with \TT{sunday} using the following: 

\hspace{-1cm}
\begin{tabular}{p{8cm} p{8cm}}
\begin{code}
Example \nm{next\_weekday\_is\_not\_saturday}: 	\\
\Forall (d:day), next\_weekday d $<>$ saturday. 	\\
\Proof.
\end{code}
&
\begin{goal}
1 subgoal			\\
\_\_\_\_\_\_\_\_\_\_\_\_\_\_\_\_\_\_\_\_\_\_\_\_\_\_\_\_\_\_\_\_\_\_\_\_\_\_(1/1)	\\
\Forall d : day, next\_weekday d $<>$ saturday
\end{goal}
\end{tabular}

\noindent
We make sure to introduce the variable \TT{d} first:

\hspace{-1cm}
\begin{tabular}{p{8cm} p{8cm}}
\begin{code}
intros d.
\end{code}
&
\begin{goal}
1 subgoal			\\
d : day			\\
\_\_\_\_\_\_\_\_\_\_\_\_\_\_\_\_\_\_\_\_\_\_\_\_\_\_\_\_\_\_\_\_\_\_\_\_\_\_(1/1)	\\ 
next\_weekday d $<>$ saturday
\end{goal}
\end{tabular}

\noindent
Now, as with the previous proof, we want to use \TT{destruct} on \TT{d}, 
then apply \TT{discriminate} to all subgoals, which we can do by writing the following 
(the use of the semicolon (i.e., \TT{t1 ; t2}) tells the proof environment to apply the tactic following it (i.e., \TT{t1}) to the result of the tactic before it (i.e., \TT{t2})): 

\hspace{-1cm}
\begin{tabular}{p{8cm} p{8cm}}
\begin{code}
destruct d; discriminate.
\end{code}
&
\begin{goal}
No more subgoals.
\end{goal}
\end{tabular}

\noindent
Finally, we close out the proof environment:

\hspace{-1cm}
\begin{tabular}{p{8cm} p{8cm}}
\begin{code}
\Qed.
\end{code}
&
\begin{msg}
next\_weekday\_is\_not\_saturday is defined
\end{msg}
\end{tabular}




