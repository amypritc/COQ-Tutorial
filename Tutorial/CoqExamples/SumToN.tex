
Please see file ``sum\_1\_to\_n.v'' to follow along with this example. 

~\\
This example defines a recursive function that takes a number n and returns the sum of the numbers from one to n (i.e., \TT{$\Sigma_{i = 0}^n\ i$}).
It then proves that for all nat numbers, 2 * sum n = (1 + n) * n.
This assertion is equivalent to sum n = ((1 + n) * n) / 2; 
however, it is much easier to prove when we do not involve division.
First, a longer proof using properties of addition and multiplication will be shown, 
then the shorter version that applies the ring tactic to take care of the application of many of these properties for us.

\begin{code}
	\Load \nm{Arith}.
	\\ \\
	\Fixpoint \nm{sum} (n : nat) : nat := 		\\ \-\ \quad
	\match n \with						\\ \-\ \quad
	$\mid$ O $=>$ 0					\\ \-\ \quad
	$\mid$ S p $=>$ (S p) + (sum p)		\\ \-\ \quad
	\End.
\end{code}

\noindent
In this example, we will use functional induction to help us successfully complete our proof over the recursive function sum.
To do this, we must first make sure to require and load \TT{FunInd}, and then define the functional induction scheme.

\begin{code}
	\cmd{Require} \nm{FunInd}.	\\
	\Load FunInd.				\\
	Functional Scheme sum\_ind := Induction for sum Sort \ty{Prop}.
\end{code}

\noindent 
Next, we will define the theorem we would like to prove. 
For this function, we want to prove that $sum\ n = \frac{(1 + n) * n}{2}$; 
however, given that division is difficult to manage when proving theorems, we will rewrite this as $2 * sum\ n = (1 + n) * n$. 

\hspace{-1cm}
\begin{tabular}{p{8cm} p{8cm}}
\begin{code}
\Theorem \nm{Thm\_sum} :	\\ \-\ \quad
\Forall n:nat,				
2 * sum n = (1 + n) * n.		\\
\Proof. 
\end{code}
&
\begin{goal}
1 subgoal															\\
\_\_\_\_\_\_\_\_\_\_\_\_\_\_\_\_\_\_\_\_\_\_\_\_\_\_\_\_\_\_\_\_\_\_\_\_\_\_\_\_\_\_\_\_\_\_\_\_\_\_(1/1)		\\
\Forall n : nat, 2 * sum n = (1 + n) * n
\end{goal}
\end{tabular}



\noindent 
As usual, we need to introduce our variable\TT{n} first:

\hspace{-1cm}
\begin{tabular}{p{8cm} p{8cm}}
\begin{code}
intros n. 
\end{code}
&
\begin{goal}
1 subgoal														\\
n : nat														\\
\_\_\_\_\_\_\_\_\_\_\_\_\_\_\_\_\_\_\_\_\_\_\_\_\_\_\_\_\_\_\_\_\_\_\_\_\_\_\_\_\_\_\_\_\_\_\_\_\_\_(1/1)	\\
2 * sum n = (1 + n) * n
\end{goal}
\end{tabular}



\noindent 
Then we can declare that we would like to use functional induction to complete this proof: 

\hspace{-1cm}
\begin{tabular}{p{8cm} p{8cm}}
\begin{code}
functional induction (sum n)	\\ \-\ \quad
  using sum\_ind.
\end{code}
&
\begin{goal}
2 subgoals													\\
\_\_\_\_\_\_\_\_\_\_\_\_\_\_\_\_\_\_\_\_\_\_\_\_\_\_\_\_\_\_\_\_\_\_\_\_\_\_\_\_\_\_\_\_\_\_\_\_\_\_(1/2)	\\
2 * 0 = (1 + 0) * 0												\\
\_\_\_\_\_\_\_\_\_\_\_\_\_\_\_\_\_\_\_\_\_\_\_\_\_\_\_\_\_\_\_\_\_\_\_\_\_\_\_\_\_\_\_\_\_\_\_\_\_\_(2/2)	\\
2 * (S p + sum p) = (1 + S p) * S p
\end{goal}
\end{tabular}



\noindent 
We enter into the first subgoal: 

\hspace{-1cm}
\begin{tabular}{p{8cm} p{8cm}}
\begin{code}
- 
\end{code}
&
\begin{goal}
1 subgoal														\\
\_\_\_\_\_\_\_\_\_\_\_\_\_\_\_\_\_\_\_\_\_\_\_\_\_\_\_\_\_\_\_\_\_\_\_\_\_\_\_\_\_\_\_\_\_\_\_\_\_\_(1/1)	\\
2 * 0 = (1 + 0) * 0
\end{goal}
\end{tabular}



\noindent 
This subgoal can be solved simply by using the tactic \TT{trivial} (this tactic will recognize that both sides will evaluate to 0):

\hspace{-1cm}
\begin{tabular}{p{8cm} p{8cm}}
\begin{code}
- trivial. 
\end{code}
&
\begin{goal}
This subproof is complete, but there are some unfocused goals:			\\
\_\_\_\_\_\_\_\_\_\_\_\_\_\_\_\_\_\_\_\_\_\_\_\_\_\_\_\_\_\_\_\_\_\_\_\_\_\_\_\_\_\_\_\_\_\_\_\_\_\_(1/1) 	\\
2 * (S p + sum p) = (1 + S p) * S p
\end{goal}
\end{tabular}



\noindent 
Now we enter into our second subgoal:

\hspace{-1cm}
\begin{tabular}{p{8cm} p{8cm}}
\begin{code}
-
\end{code}
&
\begin{goal}
1 subgoal														\\
p : nat														\\
IHn0 : 2 * sum p = (1 + p) * p										\\
\_\_\_\_\_\_\_\_\_\_\_\_\_\_\_\_\_\_\_\_\_\_\_\_\_\_\_\_\_\_\_\_\_\_\_\_\_\_\_\_\_\_\_\_\_\_\_\_\_\_(1/1)	\\
2 * (S p + sum p) = (1 + S p) * S p
\end{goal}
\end{tabular}



\noindent 
Now, we need to look at our goal and decide what to do from here 
(keep in mind that we want to be able to match some part of our goal with one of the 
	sides of the equation of the inductive hypothesis).
We can try the tactic \TT{auto}, but nothing happens. 
We can try the tactic \TT{simpl}: 

\hspace{-1cm}
\begin{tabular}{p{8cm} p{8cm}}
\begin{code}
- simpl. 
\end{code}
&
\begin{goal}
1 subgoal														\\
p : nat														\\
IHn0 : 2 * sum p = (1 + p) * p										\\
\_\_\_\_\_\_\_\_\_\_\_\_\_\_\_\_\_\_\_\_\_\_\_\_\_\_\_\_\_\_\_\_\_\_\_\_\_\_\_\_\_\_\_\_\_\_\_\_\_\_(1/1)	\\
S (p + sum p + S (p + sum p + 0)) =									\\
S (p + S (p + p * S p))
\end{goal}
\end{tabular}



\noindent 
\TT{simpl} lead to the multiplications broken down, but now there are more terms to try to reason about what we should do next, so lets go back a step again. 
To get ideas on what we could apply, we can use the command \TT{Search}, such as in \TT{Search Nat.add.}
We can see that it is possible to apply \TT{Nat.add\_succ\_comm} to the right hand side of our goal:


\hspace{-1cm}
\begin{tabular}{p{8cm} p{8cm}}
\begin{code}
rewrite $<-$ Nat.add\_succ\_comm.
\end{code}
&
\begin{goal}
1 subgoal														\\
p : nat														\\
IHn0 : 2 * sum p = (1 + p) * p										\\
\_\_\_\_\_\_\_\_\_\_\_\_\_\_\_\_\_\_\_\_\_\_\_\_\_\_\_\_\_\_\_\_\_\_\_\_\_\_\_\_\_\_\_\_\_\_\_\_\_\_(1/1)	\\
2 * (S p + sum p) = (2 + p) * S p
\end{goal}
\end{tabular}



\noindent 
This doesn't make the goal more complex, but it also doesn't particularly help us get closer to being able to use the inductive hypothesis to make substitutions in our goal. 
We can use \TT{Search Nat.mul.} to try to give us some more ideas of properties we could use to get to that point. 
When looking through the results, \TT{Nat.mul\_add\_distr\_r} and \TT{Nat.mul\_add\_distr\_l} stood out to me as being helpful, so we can try those next:

\hspace{-1cm}
\begin{tabular}{p{8cm} p{8cm}}
\begin{code}
rewrite $->$ Nat.mul\_add\_distr\_r.
\end{code}
&
\begin{goal}
1 subgoal														\\
p : nat														\\
IHn0 : 2 * sum p = (1 + p) * p										\\
\_\_\_\_\_\_\_\_\_\_\_\_\_\_\_\_\_\_\_\_\_\_\_\_\_\_\_\_\_\_\_\_\_\_\_\_\_\_\_\_\_\_\_\_\_\_\_\_\_\_(1/1)	\\
2 * (S p + sum p) = 2 * S p + p * S p
\end{goal}
\end{tabular}



\noindent 
This helped us get rid of the parentheses on the right hand side of the goal, but didn't really get us closer to substituting in the inductive hypothesis. 
Lets try \TT{Nat.mul\_add\_distr\_l} next: 

\hspace{-1cm}
\begin{tabular}{p{8cm} p{8cm}}
\begin{code}
rewrite $->$ Nat.mul\_add\_distr\_l.
\end{code}
&
\begin{goal}
1 subgoal														\\
p : nat														\\
IHn0 : 2 * sum p = (1 + p) * p										\\
\_\_\_\_\_\_\_\_\_\_\_\_\_\_\_\_\_\_\_\_\_\_\_\_\_\_\_\_\_\_\_\_\_\_\_\_\_\_\_\_\_\_\_\_\_\_\_\_\_\_(1/1)	\\
2 * S p + 2 * sum p = 2 * S p + p * S p
\end{goal}
\end{tabular}



\noindent 
Now we have the left hand side of the inductive hypothesis, \TT{2 * sum p}, in our goal, so we can use the tactic \TT{rewrite} with the inductive hypothesis \TT{IHn0}: 

\hspace{-1cm}
\begin{tabular}{p{8cm} p{8cm}}
\begin{code}
rewrite $->$ IHn0.
\end{code}
&
\begin{goal}
1 subgoal														\\
p : nat														\\
IHn0 : 2 * sum p = (1 + p) * p										\\
\_\_\_\_\_\_\_\_\_\_\_\_\_\_\_\_\_\_\_\_\_\_\_\_\_\_\_\_\_\_\_\_\_\_\_\_\_\_\_\_\_\_\_\_\_\_\_\_\_\_(1/1)	\\
2 * S p + (1 + p) * p = 2 * S p + p * S p
\end{goal}
\end{tabular}



\noindent 
We can try to apply tatic \TT{simpl} again to see if that gets us closer to matching left- and right-hand sides of the goal:

\hspace{-1cm}
\begin{tabular}{p{8cm} p{8cm}}
\begin{code}
simpl.
\end{code}
&
\begin{goal}
1 subgoal														\\
p : nat														\\
IHn0 : 2 * sum p = (1 + p) * p										\\
\_\_\_\_\_\_\_\_\_\_\_\_\_\_\_\_\_\_\_\_\_\_\_\_\_\_\_\_\_\_\_\_\_\_\_\_\_\_\_\_\_\_\_\_\_\_\_\_\_\_(1/1)	\\
S (p + S (p + 0) + (p + p * p)) =										\\
S (p + S (p + 0) + p * S p)
\end{goal}
\end{tabular}



\noindent 
Both sides are very similar now, all we need to do is rewrite our goal in some way to get \TT{(p + p * p)} to be equal to \TT{p * S p}. 
Something similar to the property we first used about addition (\TT{Nat.add\_succ\_comm}) for multiplication would be useful here. 
Looking at the results of \TT{Search Nat.mul} again can lead us to find \TT{Nat.mul\_succ\_r}:

\hspace{-1cm}
\begin{tabular}{p{8cm} p{8cm}}
\begin{code}
rewrite $->$ Nat.mul\_succ\_r.
\end{code}
&
\begin{goal}
1 subgoal														\\
p : nat														\\
IHn0 : 2 * sum p = (1 + p) * p										\\
\_\_\_\_\_\_\_\_\_\_\_\_\_\_\_\_\_\_\_\_\_\_\_\_\_\_\_\_\_\_\_\_\_\_\_\_\_\_\_\_\_\_\_\_\_\_\_\_\_\_(1/1)	\\
S (p + S (p + 0) + (p + p * p)) =										\\
S (p + S (p + 0) + (p * p + p))
\end{goal}
\end{tabular}



\noindent 
We are very close to finishing now! 
We can try to apply the tactic \TT{reflexivity}, but it throws an error because it doesn't recognize that \TT{(p + p * p)} and \TT{(p * p + p)} would be equivalent on its own. 
So we need a way to move around the terms using a commutative property; we have the property \TT{Nat.add\_comm} that would be helpful, but it doesn't match exactly to apply it without giving some extra information: 

\begin{msg}
	Nat.add\_comm: forall n m : nat, n + m = m + n
\end{msg}

\noindent
In Coq, there is a way to leverage properties like this by specifying what one of the terms would be. 
For example, here we can specify that \TT{n} should be equivalent to \TT{p * p} by doing the following: 

\hspace{-1cm}
\begin{tabular}{p{8cm} p{8cm}}
\begin{code}
rewrite Nat.add\_comm with (n := p*p).
\end{code}
&
\begin{goal}
1 subgoal														\\
p : nat														\\
IHn0 : 2 * sum p = (1 + p) * p										\\
\_\_\_\_\_\_\_\_\_\_\_\_\_\_\_\_\_\_\_\_\_\_\_\_\_\_\_\_\_\_\_\_\_\_\_\_\_\_\_\_\_\_\_\_\_\_\_\_\_\_(1/1)	\\
S (p + S (p + 0) + (p + p * p)) =										\\
S (p + S (p + 0) + (p + p * p))
\end{goal}
\end{tabular}



\noindent 
Now both sides of the equation of our goal are identical, so we can use \TT{reflexivity} to show this: 

\hspace{-1cm}
\begin{tabular}{p{8cm} p{8cm}}
\begin{code}
reflexivity.
\end{code}
&
\begin{goal}
No more subgoals.
\end{goal}
\end{tabular}



\noindent 
Our proof is complete, so we use \TT{\Qed.} to close out the proof:

\hspace{-1cm}
\begin{tabular}{p{8cm} p{8cm}}
\begin{code}
\Qed.
\end{code}
&
\begin{msg}
Thm\_sum is defined
\end{msg}
\end{tabular}



\noindent 
You may think that this seems like a bit of a long proof, and there must be a shorter version; 
however, when completing proofs, the shorter version usually isn't the most obvious one at first, 
especially if you newer to completing proofs, or even just using a specific proof assistant. 
If we look back at the first proof, the first subgoal was proven using \TT{trivial}, so we will leave that as is. 
Then we can see that the only step we needed before applying the inductive hypothesis was to rewrite the goal using \TT{Nat.mul\_add\_distr\_l}, so we will make that our first step to prove the second subgoal. 
We will apply the inductive hypothesis as the next step. 
Then, after doing a bit of research (or thinking back to the tactics section, subsection \ref{ring}) we realize that the tactic \TT{ring} will go through and apply various rules about numbers, which can be used to finish our proof after we have applied the inductive hypothesis. 

~\\ \noindent
The shorter proof is as follows:  

\hspace{-1cm}
\begin{tabular}{p{8cm} p{8cm}}
\begin{code}
\Theorem \nm{Thm\_sum\_simple} :		\\ \-\ \quad
\Forall n:nat,					
2 * sum n = (1 + n) * n.				\\
\Proof.							\\	
intros n.							\\
functional induction (sum n) 
  using sum\_ind.					\\
- trivial.							\\
- rewrite $->$ Nat.mul\_add\_distr\_l.		\\ \-\ \quad
  rewrite $->$ IHn0.					\\ \-\ \quad
  ring.							\\
\Qed.
\end{code}
&
\begin{msg}
Thm\_sum\_simple is defined
\end{msg}
\end{tabular}

\noindent 
(If you would like to see all the intermediate steps over the goal for this proof, please follow along with file ``sum\_1\_to\_n.v''.)










