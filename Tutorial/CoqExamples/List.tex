
Please see file ``list\_ex.v'' to follow along with this example. 

~\\
A simple example to demonstrate some uses of the built-in lists and list functions, and define a search function for lists of nat numbers. To use the notations and functions, you'll want to load \TT{List} and open the list\_scope so everything works properly.

\begin{code}
	\cmd{Require Import} \nm{Arith}.				\\
	\cmd{Load} List. 							\\
	\cmd{Open Scope} list\_scope.					\\
	\\
	\Fixpoint \nm{search} (l : Datatypes.list nat) (n: nat) : bool :=  	\\ \-\ \quad
	  \match l \with 								\\ \-\ \qquad
	   $\mid$ [ ] $=>$ false						\\ \-\ \qquad
	   $\mid$ hd :: tl $=>$ 						\\ \-\ \qquad\qquad
	      \If (n $=?$ hd)							\\ \-\ \qquad\qquad
	      \Then true								\\ \-\ \qquad\qquad
	      \Else search tl n							\\ \-\ \quad
	  \End.									
\end{code}

\noindent
Uses of the list functions and results:
\\~\\
Searching a list for a specific element:
	  
\hspace{-1cm}
\begin{tabular}{p{8cm} p{8cm}}
\begin{code}	Compute search [1;3;6;8] 5.		\end{code}	
&
\begin{msg}	= false     : bool					\end{msg}
\end{tabular}

\noindent	
Find the number of elements in a list:

\hspace{-1cm}
\begin{tabular}{p{8cm} p{8cm}}
\begin{code}	Compute length [1;2;3].			\end{code}
&
\begin{msg}	= 3     : nat					\end{msg}
\end{tabular}

\noindent
Get the first (head) element of a list (if there is one): 

\hspace{-1cm}
\begin{tabular}{p{8cm} p{8cm}}
\begin{code}	Compute head [2;4;6].			\end{code}
&
\begin{msg}	= Some 2     : option nat			\end{msg}
\end{tabular}

\noindent
Get the remainder (tail) of the list / remove the head element of a list: 

\hspace{-1cm}
\begin{tabular}{p{8cm} p{8cm}}
\begin{code}	Compute tail [3;6;9].				\end{code}
&
\begin{msg}	= [6; 9]     : Datatypes.list nat		\end{msg}
\end{tabular}

\noindent
Append two lists together (using the app function):

\hspace{-1cm}
\begin{tabular}{p{8cm} p{8cm}}
\begin{code}	Compute app [1;2] [3;4].			\end{code}
&
\begin{msg}	= [1; 2; 3; 4]     : Datatypes.list nat	\end{msg}
\end{tabular}

\noindent
Append two lists together (using the $++$ notation):

\hspace{-1cm}
\begin{tabular}{p{8cm} p{8cm}}
\begin{code}	Compute [1;2] ++ [3;4].			\end{code}
&
\begin{msg}	= [1; 2; 3; 4]     : Datatypes.list nat	\end{msg}
\end{tabular}

\newpage
\noindent
Add a single element to the beginning of a list:

\hspace{-1cm}
\begin{tabular}{p{8cm} p{8cm}}
\begin{code}	Compute 1::[2;3;5;7].				\end{code}
&
\begin{msg}	= [1; 2; 3; 5; 7]     : Datatypes.list nat	\end{msg}
\end{tabular}

\noindent
Reverse the elements in a list: 

\hspace{-1cm}
\begin{tabular}{p{8cm} p{8cm}}
\begin{code}	Compute rev [9;7;5;3;1].			\end{code}
&
\begin{msg}	= [1; 3; 5; 7; 9]     : Datatypes.list nat	\end{msg}
\end{tabular}

\noindent
Find the $nth$ element of a list (lists are 0-indexed): 

\hspace{-1cm}
\begin{tabular}{p{8cm} p{8cm}}
\begin{code}	Compute nth 0 [2;5;7;9].			\end{code}	
&
\begin{msg}	= fun \_ : nat $=>$ 2     : nat $->$ nat	\end{msg}
\end{tabular}

\noindent
Find the $nth$ element of a list (out of bounds): 

\hspace{-1cm}
\begin{tabular}{p{8cm} p{8cm}}
\begin{code}	Compute nth 4 [2;5;7;9].			
			\\ \cmt{Not found} 				\end{code}
&
\begin{msg}	= fun default : nat $=>$       default
		\\     : nat $->$ nat					\end{msg}
\end{tabular}

\noindent
Applying a function to each element of an array (here, we multiply each element by 3):

\hspace{-1cm}
\begin{tabular}{p{8cm} p{8cm}}
\begin{code}	Compute map (Nat.mul 3) [1;2;3].	\end{code}
&
\begin{msg}	= [3; 6; 9]     : Datatypes.list nat		\end{msg}
\end{tabular}



