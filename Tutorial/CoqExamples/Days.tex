
Please see file ``days.v'' to follow along with this example. 

~\\
This example is fairly simple, giving some basic definitions and functions, and showing a use of a tuple in Coq.  

~\\
\noindent
First, we define what a day is using the following definition:

\begin{code}
\Inductive \nm{day}: \Type :=		\\ \-\ \quad
  $\mid$ monday : day			\\ \-\ \quad
  $\mid$ tuesday : day			\\ \-\ \quad
  $\mid$ wednesday : day			\\ \-\ \quad
  $\mid$ thursday : day			\\ \-\ \quad
  $\mid$ friday : day				\\ \-\ \quad
  $\mid$ saturday : day			\\ \-\ \quad
  $\mid$ sunday : day.
\end{code}

\noindent
Then we define a simple function to compute what the next weekday is after a given day. 
This takes a day as input, and gives back what the next weekday would be.

\begin{code}
\Definition \nm{next\_weekday} (d:day): day :=		\\ \-\ \quad
  \match d \with								\\ \-\ \qquad
   $\mid$ monday $=>$ tuesday					\\ \-\ \qquad
   $\mid$ tuesday $=>$ wednesday				\\ \-\ \qquad
   $\mid$ wednesday $=>$ thursday				\\ \-\ \qquad
   $\mid$ thursday $=>$ friday					\\ \-\ \qquad
   $\mid$ friday $=>$ monday					\\ \-\ \qquad
   $\mid$ saturday $=>$ monday				\\ \-\ \qquad
   $\mid$ sunday $=>$ monday					\\ \-\ \quad
  \End.
\end{code}

\noindent
We can perform some simple computations to check the correctness of our function:

\hspace{-1cm}
\begin{tabular}{p{9cm} p{7cm}}
\begin{code} 	Compute (next\_weekday friday).	\end{code}
\begin{code}	Compute next\_weekday (next\_weekday friday).	\end{code}	
&	
\begin{msg}	     = \nm{monday}     : \nm{day}	\end{msg}
\begin{msg}	     = \nm{tuesday}     : \nm{day}		\end{msg}
\end{tabular}

\noindent
To make this a bit more interesting, we can also define some other things, such as sports:

\begin{code}
\Inductive \nm{sport}: \Type :=		\\ \-\ \quad
  $\mid$ tennis : sport			\\ \-\ \quad
  $\mid$ basketball : sport			\\ \-\ \quad
  $\mid$ baseball : sport 			\\ \-\ \quad
  $\mid$ cricket : sport			\\ \-\ \quad
  $\mid$ football : sport			\\ \-\ \quad
  $\mid$ dance : sport			\\ \-\ \quad
  $\mid$ gymnastics : sport		\\ \-\ \quad
  $\mid$ run : sport				\\ \-\ \quad
  $\mid$ rugby : sport			\\ \-\ \quad
  $\mid$ weights : sport			\\ \-\ \quad
  $\mid$ swim : sport.
\end{code}

and activities: 

\begin{code}
\Inductive \nm{activity}: \Type :=	\\ \-\ \quad
  $\mid$ class : activity			\\ \-\ \quad
  $\mid$ lab : activity				\\ \-\ \quad
  $\mid$ meeting : activity			\\ \-\ \quad
  $\mid$ seminar : activity			\\ \-\ \quad
  $\mid$ volunteer : activity		\\ \-\ \quad
  $\mid$ club : activity			\\ \-\ \quad
  $\mid$ work : activity.	
\end{code}

\noindent
Now, it is possible to create a function that returns a schedule for each day, 
consisting of a tuple of an activity and a sport. 

\begin{code}
\Definition \nm{daily\_schedule} (d:day): activity * sport :=		\\ \-\ \quad
  \match d \with										\\ \-\ \qquad
   $\mid$ monday $=>$ (meeting, run)					\\ \-\ \qquad
   $\mid$ tuesday $=>$ (class, tennis)					\\ \-\ \qquad
   $\mid$ wednesday $=>$ (seminar, baseball)				\\ \-\ \qquad
   $\mid$ thursday $=>$ (class, rugby)					\\ \-\ \qquad
   $\mid$ friday $=>$ (lab, dance)						\\ \-\ \qquad
   $\mid$ saturday $=>$ (club, swim)						\\ \-\ \qquad
   $\mid$ sunday $=>$ (volunteer, basketball)				\\ \-\ \quad
  \End.	
\end{code}

\noindent
Of course, it is also possible to create more complex schedules, but this 
should give you an the basic idea of how to do that. 


% BEGIN  COMMENT
\begin{comment}
\noindent
We can also do a simple proof to check that the results are as expected.

\hspace{-1cm}
\begin{tabular}{p{8cm} p{8cm}}
\begin{code}
	\Example \nm{test\_next\_weekday}:			\\ \-\ \quad
	  (next\_weekday (next\_weekday saturday)) 	\\ \-\ \qquad
	  	= tuesday.							\\
	\Proof.
\end{code}
&
\begin{msg}
1 subgoal			\\
\_\_\_\_\_\_\_\_\_\_\_\_\_\_\_\_\_\_\_\_\_\_\_\_\_\_\_\_\_\_\_\_\_\_\_\_\_\_(1/1)	\\
next\_weekday (next\_weekday saturday) = tuesday
\end{msg}
\end{tabular}
\end{comment}
% END  COMMENT






