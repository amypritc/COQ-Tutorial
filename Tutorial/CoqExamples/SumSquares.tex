
Please see file ``sumsq.v'' to follow along with this example. 

~\\
In this example, we define the function \TT{sumsq}, or the sum of squares 
(i.e., \TT{$\Sigma_{i=0}^n\ (i * i)$}). 


\hspace{-1cm}
\begin{tabular}{p{8cm} p{8cm}}
\begin{code}
\Load Arith.
\\ \\ 
\Fixpoint \nm{sumsq} (n: nat) : nat := 		\\ \-\ \quad
	\match n \with						\\ \-\ \qquad
	$\mid$ O $=>$ 0					\\ \-\ \qquad
	$\mid$ S p $=>$ (n * n) + (sumsq p)		\\ \-\ \quad
	\End.
\end{code}
&
\begin{msg}
sumsq is defined						\\
sumsq is recursively defined (decreasing on 1st argument)
\end{msg}
\end{tabular}


\noindent
We then want to prove that \TT{6 * sumsq n = n * (n + 1) * (2n + 1)} 
(or \TT{sumsq n = $\frac{n * (n + 1) * (2n + 1)}{6}$}). 
To do this, we will use functional induction, so we need the following: 

\hspace{-1cm}
\begin{tabular}{p{8cm} p{8cm}}
\begin{code}
\cmd{Require} FunInd.				\\
\cmd{Load} FunInd.					\\
Functional Scheme sumsq\_ind := 		\\ \-\ \quad
  Induction \kw{for} sumsq Sort \ty{Prop}.
\end{code}
&
\begin{msg}
sumsq\_equation is defined			\\
sumsq\_ind is defined
\end{msg}
\end{tabular}



\noindent
Now we will define our theorem, \TT{Thm\_sumsq}:

\hspace{-1cm}
\begin{tabular}{p{8cm} p{8cm}}
\begin{code}
\Theorem \nm{Thm\_sumsq} :			\\ \-\ \quad
\Forall n:nat,						\\ \-\ \quad
6 * sumsq n = n * (n + 1) * (2 * n + 1).	\\
\Proof.
\end{code}
&
\begin{goal}
1 subgoal															\\
\_\_\_\_\_\_\_\_\_\_\_\_\_\_\_\_\_\_\_\_\_\_\_\_\_\_\_\_\_\_\_\_\_\_\_\_\_\_\_\_\_\_\_\_\_\_\_\_\_\_(1/1)		\\
\Forall n : nat,														\\
6 * sumsq n = n * (n + 1) * (2 * n + 1)
\end{goal}
\end{tabular}





\noindent
Then we introduce our variable \TT{n}: 

\hspace{-1cm}
\begin{tabular}{p{8cm} p{8cm}}
\begin{code}
intros n.
\end{code}
&
\begin{goal}
1 subgoal															\\
n : nat															\\
\_\_\_\_\_\_\_\_\_\_\_\_\_\_\_\_\_\_\_\_\_\_\_\_\_\_\_\_\_\_\_\_\_\_\_\_\_\_\_\_\_\_\_\_\_\_\_\_\_\_(1/1)		\\
6 * sumsq n = n * (n + 1) * (2 * n + 1)
\end{goal}
\end{tabular}






\noindent
We state we want to do functional induction using our previously defined functional induction scheme: 

\hspace{-1cm}
\begin{tabular}{p{8cm} p{8cm}}
\begin{code}
functional induction (sumsq n) 	\\ \-\ \quad
  using sumsq\_ind.
\end{code}
&
\begin{goal}
2 subgoals													\\
\_\_\_\_\_\_\_\_\_\_\_\_\_\_\_\_\_\_\_\_\_\_\_\_\_\_\_\_\_\_\_\_\_\_\_\_\_\_\_\_\_\_\_\_\_\_\_\_\_\_(1/2)	\\
6 * 0 = 0 * (0 + 1) * (2 * 0 + 1)										\\
\_\_\_\_\_\_\_\_\_\_\_\_\_\_\_\_\_\_\_\_\_\_\_\_\_\_\_\_\_\_\_\_\_\_\_\_\_\_\_\_\_\_\_\_\_\_\_\_\_\_(2/2)	\\
6 * (S p * S p + sumsq p) =										\\
S p * (S p + 1) * (2 * S p + 1)
\end{goal}
\end{tabular}





\noindent
Then we enter into the first subgoal:

\hspace{-1cm}
\begin{tabular}{p{8cm} p{8cm}}
\begin{code}
- 
\end{code}
&
\begin{goal}
1 subgoal														\\
\_\_\_\_\_\_\_\_\_\_\_\_\_\_\_\_\_\_\_\_\_\_\_\_\_\_\_\_\_\_\_\_\_\_\_\_\_\_\_\_\_\_\_\_\_\_\_\_\_\_(1/1)	\\
6 * 0 = 0 * (0 + 1) * (2 * 0 + 1)
\end{goal}
\end{tabular}





\noindent
This can be simply proven using tactic \TT{trivial}:

\hspace{-1cm}
\begin{tabular}{p{8cm} p{8cm}}
\begin{code}
- trivial.
\end{code}
&
\begin{goal}
This subproof is complete, but there are some unfocused goals:			\\

\_\_\_\_\_\_\_\_\_\_\_\_\_\_\_\_\_\_\_\_\_\_\_\_\_\_\_\_\_\_\_\_\_\_\_\_\_\_\_\_\_\_\_\_\_\_\_\_\_\_(1/1)	\\
6 * (S p * S p + sumsq p) =										\\
S p * (S p + 1) * (2 * S p + 1)
\end{goal}
\end{tabular}





\noindent
We enter into the second subgoal: 

\hspace{-1cm}
\begin{tabular}{p{8cm} p{8cm}}
\begin{code}
- 
\end{code}
&
\begin{goal}
1 subgoal														\\
p : nat														\\
IHn0 : 6 * sumsq p =												\\ \-\ \qquad\quad 
       p * (p + 1) * (2 * p + 1)										\\\_\_\_\_\_\_\_\_\_\_\_\_\_\_\_\_\_\_\_\_\_\_\_\_\_\_\_\_\_\_\_\_\_\_\_\_\_\_\_\_\_\_\_\_\_\_\_\_\_\_(1/1)	\\
6 * (S p * S p + sumsq p) =										\\
S p * (S p + 1) * (2 * S p + 1)
\end{goal}
\end{tabular}




\noindent
As with our previous proof for \TT{sum to n}, we will use \TT{Nat.mul\_add\_distr\_l}, which is defined as: 

\begin{msg}
Nat.mul\_add\_distr\_l:
  \Forall n m p : nat,
  n * (m + p) = n * m + n * p
\end{msg}

\noindent
To match this to our goal (i.e., the left-hand side, \TT{6 * (S p * S p + sumsq p)}), we have 
\TT{n := 6}, \TT{m := S p * S p}, and \TT{p := sumsq p}. 
We can specify how to exactly how Coq should rewrite it using the following: 

\hspace{-1cm}
\begin{tabular}{p{8cm} p{8cm}}
\begin{code}
- rewrite Nat.mul\_add\_distr\_l 		\\ \-\ \qquad
    \kw{with} (m := S p * S p) (n := 6)  \\ \-\ \qquad\qquad \-\
    	(p := sumsq p).
\end{code}
&
\begin{goal}
1 subgoal														\\
p : nat														\\
IHn0 : 6 * sumsq p =												\\ \-\ \qquad\quad 
       p * (p + 1) * (2 * p + 1)										\\\_\_\_\_\_\_\_\_\_\_\_\_\_\_\_\_\_\_\_\_\_\_\_\_\_\_\_\_\_\_\_\_\_\_\_\_\_\_\_\_\_\_\_\_\_\_\_\_\_\_(1/1)	\\
6 * (S p * S p) + 6 * sumsq p =										\\
S p * (S p + 1) * (2 * S p + 1)
\end{goal}
\end{tabular}




\noindent
Turns out, we could've simply specified the value we needed \TT{m} to be 
(i.e., \TT{rewrite Nat.mul\_add\_distr\_l \kw{with} (m := S p * S p).}), but the long version works fine too and makes it more clear what substitutions are desired. 
Returning to our proof, we can see that the left-hand side of our inductive hypothesis (\TT{6 * sumsq p}) is now present in the left-hand side of our goal, so we can use tactic \TT{rewrite} with the inductive hypothesis:

\hspace{-1cm}
\begin{tabular}{p{8cm} p{8cm}}
\begin{code}
rewrite $->$ IHn0.
\end{code}
&
\begin{goal}
1 subgoal														\\
p : nat														\\
IHn0 : 6 * sumsq p =												\\ \-\ \qquad\quad 
       p * (p + 1) * (2 * p + 1)										\\
\_\_\_\_\_\_\_\_\_\_\_\_\_\_\_\_\_\_\_\_\_\_\_\_\_\_\_\_\_\_\_\_\_\_\_\_\_\_\_\_\_\_\_\_\_\_\_\_\_\_(1/1)	\\
6 * (S p * S p) +
p * (p + 1) * (2 * p + 1) =											\\
S p * (S p + 1) * (2 * S p + 1)
\end{goal}
\end{tabular}




\noindent
Now that we've removed the use of the function \TT{sumsq}, we can try tactic \TT{ring} to progress the proof: 

\hspace{-1cm}
\begin{tabular}{p{8cm} p{8cm}}
\begin{code}
ring. 
\end{code}
&
\begin{goal}
No more subgoals.
\end{goal}
\end{tabular}





\noindent
As you can see, \TT{ring} successfully proved the remainder of that subgoal. 
Since there are no subgoals remaining, we finish our proof: 

\hspace{-1cm}
\begin{tabular}{p{8cm} p{8cm}}
\begin{code}
\Qed. 
\end{code}
&
\begin{msg}
Thm\_sumsq is defined
\end{msg}
\end{tabular}



















